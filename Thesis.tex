% !TeX encoding = UTF-8
% !TeX program = pdflatex
% !TeX spellcheck = en_US
\documentclass[binding=0.6cm]{sapthesis}

% ============================================================================
% PACKAGES
% ============================================================================
\usepackage{microtype}
\usepackage[english]{babel}
\usepackage[utf8]{inputenc}
\usepackage{tikz}
\usepackage{float}
\usepackage{amsmath}
\usetikzlibrary{positioning, arrows.meta}

% Mathematical packages
\usepackage{amsmath}
\usepackage{amssymb}
\usepackage{amsthm}

% Graphics and figures
\usepackage{graphicx}
\usepackage{float}
\usepackage{subcaption}
\usepackage{tikz}
\usepackage{pgfplots}
\pgfplotsset{compat=1.18}

% Tables
\usepackage{booktabs}
\usepackage{array}
\usepackage{tabularx}
\usepackage{longtable}
\usepackage{multirow}

% Code listings
\usepackage{listings}
\usepackage{xcolor}

% Bibliography
\usepackage[backend=biber,style=ieee,sorting=none]{biblatex}
\addbibresource{references.bib}

% Hyperlinks (load last)
\usepackage{hyperref}
\hypersetup{
    pdftitle={A Standardized Framework for Cyber Risk Assessment Across the Lifecycle of Space Projects: Methodology and Automated Tool Development},
    pdfauthor={Giuseppe Nonni},
    colorlinks=true,
    linkcolor=blue,
    citecolor=red,
    urlcolor=blue
}

% Code style configuration
\lstset{
    backgroundcolor=\color{gray!10},
    basicstyle=\ttfamily\small,
    breaklines=true,
    captionpos=b,
    commentstyle=\color{green!50!black},
    frame=single,
    keywordstyle=\color{blue},
    language=Python,
    numbers=left,
    numberstyle=\tiny\color{gray},
    showspaces=false,
    showstringspaces=false,
    stringstyle=\color{orange},
    tabsize=2
}

% ============================================================================
% THESIS INFORMATION
% ============================================================================
\title{A Standardized Framework for Cyber Risk Assessment Across the Lifecycle of Space Projects: Methodology and Automated Tool Development}
\examdate{12/12/12}

\author{Giuseppe Nonni}
\IDnumber{1948023}
\course{Cybersecurity}
\courseorganizer{Information Engineering, Informatics, and Statistics}
\AcademicYear{2024/2025}
\advisor{Prof. Marco Angelini}
\coadvisor{Ing. Giorgio Sciascia}
\customcoadvisorlabel{Company Supervisor}
\authoremail{nonni.1948023@studenti.uniroma1.it}
\copyyear{2025}
\thesistype{Master thesis}

% ============================================================================
% DOCUMENT BEGIN
% ============================================================================
\begin{document}

\frontmatter
\maketitle

\dedication{Dedicated to\\
all the space explorers\\
who dare to reach for the stars}

\begin{abstract}
Risk assessment is a fundamental component of space mission planning and operation, ensuring the identification, analysis, and mitigation of threats across all project phases. However, current risk assessment methodologies often lack standardization, leading to inefficiencies and inconsistencies, particularly when comparing risks across different projects. This thesis proposes a comprehensive framework for standardized risk assessment applicable to all phases of a space project, from the proposal phase to operational maintenance and end-of-life management.

The proposed framework establishes a structured methodology that allows for systematic risk identification, assessment and monitoring across multiple projects, reducing redundancy and improving comparability. Additionally, this research includes the development of an automated tool suite that simplifies risk evaluation processes, enabling efficient and repeatable assessments, facilitating objective risk comparisons between projects and supporting decision-making and resource allocation.

The developed Risk Assessment Tool Suite consists of four integrated components: a BID Phase assessment tool for initial project categorization, a preliminary risk assessment module for early-stage threat identification, a comprehensive risk evaluation system with detailed criteria analysis, and an attack graph analyzer for visualizing threat relationships in space systems. Each tool follows consistent methodological principles while addressing specific aspects of the risk assessment lifecycle.

Through this approach, the thesis aims to enhance risk governance in the space sector, ensuring consistency, efficiency, and traceability in risk management activities. The framework and toolset developed contribute to a more resilient and predictable project lifecycle, ultimately improving security during all phases of the mission and its cost-effectiveness. The tools have been validated through practical implementation and demonstrate significant improvements in assessment accuracy, time efficiency, and standardization compared to traditional manual approaches.
\end{abstract}

\tableofcontents
\listoffigures
\listoftables

\mainmatter

% ============================================================================
% CHAPTER 1: INTRODUCTION
% ============================================================================
\chapter{Introduction}
\label{ch:introduction}

The space industry has experienced unprecedented growth in recent decades, with missions becoming increasingly complex and interconnected. From satellite constellations providing global communications to deep space exploration missions, the cybersecurity challenges facing space systems have evolved dramatically. Traditional risk assessment methodologies, often developed for terrestrial systems, prove inadequate when applied to the unique constraints and threat landscape of space missions.

This thesis addresses the critical need for standardized cybersecurity risk assessment frameworks specifically designed for space projects. The research presents both theoretical foundations and practical implementations through the development of an integrated Risk Assessment Tool Suite.

\section{Problem Statement}

Current risk assessment practices in the space domain suffer from several critical limitations:

\begin{itemize}
    \item \textbf{Lack of Standardization}: Different organizations and projects employ varying methodologies, making cross-project comparisons and lessons learned difficult to apply.
    \item \textbf{Phase-Specific Gaps}: Risk assessment approaches often focus on individual project phases without considering the full lifecycle from conception to end-of-life.
    \item \textbf{Manual Processes}: Traditional assessment methods rely heavily on manual processes, leading to inconsistencies and time-intensive evaluations.
    \item \textbf{Limited Threat Visibility}: Existing frameworks often fail to capture the complex interdependencies between threats in space systems.
\end{itemize}

\section{Research Objectives}

This research aims to address these challenges through the following objectives:

\begin{enumerate}
    \item Develop a comprehensive framework for standardized cybersecurity risk assessment across all phases of space project lifecycles.
    \item Create an integrated tool suite that automates and standardizes risk evaluation processes.
    \item Establish methodologies for threat relationship analysis and attack path visualization specific to space systems.
    \item Validate the framework and tools through practical implementation and case studies.
    \item Provide guidelines for integration into existing space project management processes.
\end{enumerate}

\section{Methodology}

The research methodology combines theoretical framework development with practical tool implementation:

\begin{itemize}
    \item \textbf{Literature Review}: Comprehensive analysis of existing risk assessment frameworks, cybersecurity standards, and space-specific security challenges.
    \item \textbf{Framework Design}: Development of standardized methodologies based on industry best practices and space domain requirements.
    \item \textbf{Tool Development}: Implementation of software tools that embody the framework principles and provide automated assessment capabilities.
    \item \textbf{Validation}: Testing and refinement of tools through practical application scenarios.
\end{itemize}

\section{Contributions}

This thesis makes several key contributions to the field of space cybersecurity:

\begin{enumerate}
    \item A novel standardized framework for lifecycle-based risk assessment in space projects.
    \item An integrated software tool suite providing automated risk evaluation capabilities.
    \item Methodologies for attack graph analysis tailored to space system architectures.
    \item Validation of the framework through practical implementation and testing.
    \item Guidelines for industry adoption and integration into existing processes.
\end{enumerate}

\section{Thesis Structure}

The remainder of this thesis is organized as follows:

\textbf{Chapter 2} provides background information on space systems security, existing risk assessment frameworks, and related work in the field.

\textbf{Chapter 3} presents the theoretical foundation of the proposed standardized framework, including its underlying principles and methodological approach.

\textbf{Chapter 4} details the design and implementation of the Risk Assessment Tool Suite, including architecture decisions and technical specifications.

\textbf{Chapter 5} describes the individual components of the tool suite: BID Phase assessment, preliminary risk evaluation, comprehensive risk assessment with dynamic cybersecurity controls management, and attack graph analysis.

\textbf{Chapter 6} presents the validation methodology and results from comprehensive testing of the framework and tools, including technical validation and lessons learned.

\textbf{Chapter 7} provides comprehensive user manual and operational guidance for the Risk Assessment Tool Suite, including detailed instructions for all four integrated tools, advanced Controls Management system with real-time impact analysis, interface descriptions, and best practices for effective utilization in professional environments.

\textbf{Chapter 8} discusses the AI training methodologies and development approaches used in the automated components of the Risk Assessment Tool Suite.

\textbf{Chapter 9} discusses the implications of the research, technical challenges resolved, limitations of the current approach, and directions for future work.

\textbf{Chapter 10} concludes the thesis with a summary of contributions and their significance for the space cybersecurity domain.

% ============================================================================
% CHAPTER 2: BACKGROUND AND RELATED WORK
% ============================================================================
\chapter{Background and Related Work}
\label{ch:background}

\section{Space Systems Security Landscape}

Space systems present unique cybersecurity challenges that distinguish them from terrestrial information systems. The operating environment, communication constraints, and mission-critical nature of space assets create a complex threat landscape requiring specialized security approaches.

\subsection{Space System Architecture}

Modern space missions typically consist of four main segments:

\begin{itemize}
    \item \textbf{Space Segment}: Satellites, spacecraft, and orbital infrastructure
    \item \textbf{Ground Segment}: Ground stations, mission control centers, and data processing facilities
    \item \textbf{Link Segment}: Communication channels and data transmission pathways that interconnect all other segments, serving as the critical backbone for command, control, and data exchange between space and terrestrial assets
    \item \textbf{User Segment}: End-user terminals and applications consuming space-based services
\end{itemize}

Each segment presents distinct security challenges and threat vectors that must be considered in comprehensive risk assessment frameworks. The link segment is particularly critical as it represents the primary attack surface for adversaries seeking to intercept communications, inject malicious commands, or disrupt mission operations through communication interference. Since all mission-critical data and control signals must traverse these communication pathways, the link segment often becomes the most vulnerable point in the entire system architecture, requiring specialized security measures such as encryption, authentication protocols, and anti-jamming techniques.

\subsection{Threat Landscape}

The threat landscape for space systems encompasses both traditional cybersecurity threats and space-specific attack vectors, as extensively documented in CCSDS security recommendations and the ENISA Space Threat Landscape analysis. The unique operational environment of space systems creates vulnerability patterns that differ significantly from terrestrial IT infrastructure, requiring specialized threat modeling approaches.

\subsubsection{CCSDS-Identified Threat Categories}

The Consultative Committee for Space Data Systems (CCSDS) has identified several critical threat categories that specifically target space mission components:

\begin{itemize}
    \item \textbf{Data Corruption}: Intentional or accidental modification of mission-critical data during transmission or storage
    \item \textbf{Physical/Logical Attacks}: Direct targeting of space assets through kinetic means or sophisticated cyber intrusion techniques
    \item \textbf{Interception/Eavesdropping}: Unauthorized monitoring of communication channels to extract sensitive mission data or operational intelligence
    \item \textbf{Jamming}: Deliberate interference with radio frequency communications to disrupt command and control operations
    \item \textbf{Denial-of-Service}: Overwhelming system resources to prevent legitimate access to critical space services
    \item \textbf{Masquerade/Spoofing}: Impersonation attacks targeting command authentication systems and ground station interfaces
    \item \textbf{Replay Attacks}: Retransmission of previously captured legitimate commands to trigger unauthorized spacecraft operations
    \item \textbf{Software Threats}: Malicious code injection, firmware manipulation, and exploitation of software vulnerabilities in space-qualified systems
    \item \textbf{Unauthorized Access/Hijacking}: Complete compromise of spacecraft control systems or ground segment infrastructure
    \item \textbf{Tainted Hardware Components}: Supply chain compromise involving malicious modifications to space-qualified hardware
    \item \textbf{Supply Chain Vulnerabilities}: Risks introduced through third-party vendors, contractors, and international partnerships
\end{itemize}

\subsubsection{ENISA Space Threat Assessment}

The European Union Agency for Cybersecurity (ENISA) Space Threat Landscape provides additional insights into emerging risks facing the European space sector. Key findings include:

\begin{itemize}
    \item \textbf{Nefarious Activity/Abuse (NAA)}: Intended actions that target ICT systems, infrastructure and network by means of malicious acts with the aim to either steal, alter or destroy a specified target
    \item \textbf{Eavesdropping/Interception/Hijacking (EIH)}: Actions aiming to listen, interrupt or seize control of a third party communication without consent
    \item \textbf{Physical Attacks (PA)}: Actions which aim to destroy, exposure, alter, disable, steal or gain unauthorised access to a physical asset such infrastructure, hardware or interconnection
    \item \textbf{Unintentional Damage (UD)}: Unintentional actions causing desctruction, harm or injury of property or persons and results in a failure or reduction in usefulness
    \item \textbf{Failures or malfunctions (FM)}: Partial or full insufficient functioning of an asset (hardware or software) 
    \item \textbf{Outages (OUT)}: Unexpected disruption of service or decrease in quality falling below a required level 
    \item \textbf{Disaster (DIS)}: A sudden accident or a natural catastrophe that causes great damage or loss of life
    \item \textbf{Legal (LEG)}: Legal actions of third parties (contracting or otherwise) in order to prohibit actions or compensate for loss based on applicable law
\end{itemize}


\subsection{Asset Landscape}

Understanding the comprehensive asset landscape is crucial for effective threat assessment in space systems. The Risk Assessment Tool Suite incorporates a detailed asset taxonomy derived from extensive analysis of space mission architectures, providing a foundation for systematic vulnerability identification and risk evaluation.

\paragraph{Ground Segment Assets}

The ground segment represents the terrestrial foundation of space operations, encompassing critical infrastructure that enables mission control, data processing, and user interface capabilities:

\begin{itemize}
    \item \textbf{Ground Stations}: Essential components for space-ground communications including tracking systems for spacecraft position monitoring, ranging equipment for distance measurement and orbital determination, transmission systems for uplink command delivery, and reception systems for downlink data acquisition
    \item \textbf{Mission Control Centers}: Centralized facilities managing telemetry processing for real-time spacecraft health monitoring, commanding systems for operational control, and analysis and support infrastructure for mission decision-making
    \item \textbf{Data Processing Centers}: Specialized facilities handling mission analysis for scientific and operational data interpretation, and payload processing systems for instrument data calibration and distribution
    \item \textbf{Remote Terminals}: Distributed access points providing network access for remote operations and software access for distributed mission management
    \item \textbf{User Ground Segment}: Supporting infrastructure including development systems for mission software and procedures, supportive infrastructure for mission planning and analysis, and operational systems for routine mission execution
\end{itemize}

\paragraph{Space Segment Assets}

The space segment comprises the orbital components that execute mission objectives in the harsh environment of space:

\begin{itemize}
    \item \textbf{Platform Subsystems}: Core spacecraft infrastructure including electrical power systems for energy generation and distribution, attitude control systems for spacecraft orientation and stability, communication systems for space-ground data links, command and data handling systems for onboard processing, telemetry systems for spacecraft health monitoring, and tracking systems for navigation and positioning
    \item \textbf{Payload Systems}: Mission-specific components including payload data handling systems for scientific or commercial data processing, payload communication modules for specialized data transmission, and untrusted data handling systems for processing external or unverified data sources
\end{itemize}

\paragraph{Link Segment Assets}

The link segment provides the critical communication pathways that enable coordination and data exchange across the entire space system:

\begin{itemize}
    \item \textbf{Inter-System Links}: Communication channels between platform and payload subsystems, enabling coordinated spacecraft operations
    \item \textbf{Intra-Segment Links}: Connections within ground segment components, between multiple space systems for constellation operations, and between ground wide area networks for distributed operations
    \item \textbf{Cross-Segment Links}: Critical pathways connecting space segment to ground segment for mission control, space segment to user segment for service delivery, and ground segment to user segment for data distribution
    \item \textbf{User-to-User Links}: Communication channels enabling inter-user connectivity and collaborative operations
\end{itemize}

\paragraph{User Segment Assets}

The user segment encompasses the end-user infrastructure that consumes and processes space-based services:

\begin{itemize}
    \item \textbf{User Communications}: Transmission systems for uplink communications to space systems and reception systems for downlink data acquisition
    \item \textbf{Processing Infrastructure}: Local processing capabilities for user-specific data analysis and application execution
\end{itemize}

This comprehensive asset landscape provides the foundation for systematic threat modeling and risk assessment across all phases of space mission lifecycles. By understanding the complete inventory of critical assets and their interdependencies, security professionals can develop more effective threat scenarios and implement appropriate protective measures tailored to the specific characteristics of each asset category.

\vspace{13pt}
The evolving landscape requires continuous adaptation of security measures and risk assessment methodologies to address both current and emerging risks. This dynamic environment reinforces the importance of standardized, comprehensive risk assessment frameworks that can accommodate the unique challenges facing space systems across all mission phases.

\section{Existing Risk Assessment Frameworks}

\subsection{Traditional IT Risk Assessment}

Traditional IT risk assessment standards, such as ISO/IEC 27005 or NIST SP 800-30, are not fully adequate for conducting risk analysis in space programs due to the unique characteristics of the space domain. These standards are primarily designed for terrestrial IT environments and often fail to account for critical space-specific factors such as the harsh physical environment (radiation, temperature extremes, microgravity), the limited ability to perform physical maintenance or incident response in orbit, the high dependency on long-distance and delay-sensitive communication links, and the criticality of real-time operations. Moreover, the asset taxonomy, threat landscape, and impact evaluation criteria used in space missions differ significantly from conventional IT systems. As a result, applying traditional IT frameworks without significant adaptation may lead to underestimation of mission-critical risks or overlook threats unique to the space ecosystem.

\subsubsection{ISO/IEC 27005 Information Security Risk Management}

ISO/IEC 27005 provides a comprehensive approach to information security risk management but requires significant adaptation for space applications. The standard's strength lies in its systematic methodology for risk identification, analysis, and treatment. However, its terrestrial focus limits applicability to space systems in several key areas:

\begin{itemize}
    \item \textbf{Asset Classification}: The standard's asset taxonomy does not adequately address space-specific components such as spacecraft subsystems, orbital mechanics considerations, or ground-space communication links
    \item \textbf{Threat Environment}: Limited consideration of space-specific threats such as radiation-induced errors, orbital debris impacts, or space weather effects
    \item \textbf{Vulnerability Assessment}: Insufficient attention to unique space system vulnerabilities including long communication delays, limited bandwidth, and inability to perform physical maintenance
    \item \textbf{Impact Analysis}: Terrestrial impact categories do not fully capture the consequences of space mission failures, including cascading effects on dependent services and scientific objectives
\end{itemize}

\subsubsection{NIST SP 800-30 Guide for Conducting Risk Assessments}

The NIST Risk Management Framework provides structured methodologies for conducting risk assessments within the broader context of organizational risk management. While comprehensive in scope, several limitations emerge when applied to space systems:

\begin{table}[H]
\centering
\caption{NIST SP 800-30 Adaptation Challenges for Space Systems}
\begin{tabular}{|p{3.5cm}|p{4cm}|p{6cm}|}
\hline
\textbf{Framework Component} & \textbf{Standard Approach} & \textbf{Space Domain Challenges} \\ \hline
Risk Assessment Process & Sequential, iterative  analysis & Requires parallel assessment across segments \\ \hline
Threat Source Identification & IT-centric threat actors & Must include space-specific threats \\ \hline
Vulnerability Assessment & Software/network focus & Hardware, environmental vulnerabilities \\ \hline
Likelihood Determination & Historical data emphasis & Limited space incident data \\ \hline
Impact Analysis & Confidentiality, integrity, availability & Mission success, safety considerations \\ \hline
\end{tabular}
\end{table}

\subsection{Space-Specific Standards}

Industry standards such as CCSDS (Consultative Committee for Space Data Systems) security recommendations and ECSS (European Cooperation for Space Standardization) provide space-specific guidance that addresses the unique technical, operational, and environmental characteristics of space missions. These standards are crucial for defining consistent practices in areas such as data handling, ground-space communication protocols, system reliability, and interface specifications. However, while they offer important security and system design recommendations tailored to space systems, they often lack detailed and systematic methodologies for performing comprehensive cyber and physical risk assessments. In particular, they do not always include structured threat modeling techniques, quantitative risk metrics, or formalized impact-probability matrices adapted to the dynamic and interdependent nature of space assets. Consequently, organizations implementing these standards must often integrate additional frameworks or develop custom risk assessment processes to evaluate vulnerabilities, threats, and mitigation strategies effectively across all mission phases. This gap highlights the need for harmonizing space-domain expertise with advanced risk assessment methodologies to ensure robust protection of both space and ground segments.

\subsubsection{CCSDS Security Framework}

The Consultative Committee for Space Data Systems (CCSDS) has developed several security-related recommendations that provide foundational guidance for space mission security:

\begin{itemize}
    \item \textbf{CCSDS 350.0-G-3}: Space Communications Security Guidelines, which establish baseline security requirements for space-ground communications
    \item \textbf{CCSDS 351.0-B-2}: Space Data Systems Security Specification, defining cryptographic algorithms and protocols for space applications
    \item \textbf{CCSDS 352.0-B-2}: Space Communications Cross Support Security Specification, addressing multi-agency cooperation scenarios
\end{itemize}

While these standards provide essential technical specifications, they primarily focus on communication security rather than comprehensive risk assessment methodologies.

\subsubsection{ECSS Standards for Space Systems}

The European Cooperation for Space Standardization (ECSS) provides a comprehensive framework for space system development, including security considerations:

\begin{itemize}
    \item \textbf{ECSS-Q-ST-80}: Software Product Assurance, addressing software security throughout the development lifecycle
    \item \textbf{ECSS-E-ST-40}: Software Engineering, including security requirements and verification approaches
    \item \textbf{ECSS-M-ST-10}: Space Project Management, incorporating security considerations into project planning
\end{itemize}

These standards provide systematic approaches to space system development but require supplementation with dedicated risk assessment methodologies.

\section{Related Work}

Recent research in space cybersecurity has addressed various aspects of the security challenge, but few comprehensive frameworks exist for standardized risk assessment across project lifecycles. The academic and industrial communities have made significant contributions to understanding space system vulnerabilities and developing targeted security solutions, yet gaps remain in systematic risk assessment methodologies.

\subsection{Academic Research Contributions}

\subsubsection{Space System Vulnerability Analysis}

Fundamental research has identified critical vulnerability patterns in space systems:

\begin{itemize}
    \item \textbf{Communication Security}: Pavur et al. (2020) demonstrated vulnerabilities in satellite communication protocols, highlighting the need for enhanced encryption and authentication mechanisms
    \item \textbf{Ground Segment Security}: Santamarta (2014) revealed widespread vulnerabilities in satellite ground infrastructure, emphasizing the importance of terrestrial security measures
    \item \textbf{Physical Layer Attacks}: Investigation of RF jamming, spoofing, and interception techniques targeting space communication links
\end{itemize}

\subsubsection{Threat Modeling Methodologies}

Several researchers have developed specialized threat modeling approaches for space systems:

\begin{itemize}
    \item \textbf{Multi-Domain Threat Models}: Frameworks addressing threats across space, cyber, and terrestrial domains simultaneously
    \item \textbf{Attack Graph Analysis}: Graph-theoretic approaches for modeling complex attack scenarios in distributed space systems
    \item \textbf{Scenario-Based Assessment}: Risk assessment methodologies based on realistic attack scenarios derived from threat intelligence
\end{itemize}

\subsection{Industry Initiatives and Standards Development}

\subsubsection{Commercial Space Security}

The commercial space industry has driven several important security initiatives:

\begin{itemize}
    \item \textbf{Satellite Industry Association}: Development of cybersecurity best practices for commercial satellite operations
    \item \textbf{Space Information Sharing and Analysis Center (Space ISAC)}: Industry collaboration platform for sharing threat intelligence and security practices
    \item \textbf{Commercial Space Transportation Security}: FAA and industry collaboration on securing commercial launch and space transportation systems
\end{itemize}

\subsubsection{International Cooperation Efforts}

Global efforts to address space security challenges include:

\begin{itemize}
    \item \textbf{United Nations Office for Outer Space Affairs (UNOOSA)}: Development of international guidelines for space system security
    \item \textbf{Inter-Agency Space Debris Coordination Committee (IADC)}: Collaboration on space debris mitigation with security implications
    \item \textbf{European Space Agency Security Office}: Development of European-wide space security policies and standards
\end{itemize}

\subsection{Research Gaps and Limitations}

Despite significant progress, several critical gaps remain in space cybersecurity research:

\begin{table}[H]
\centering
\caption{Current Research Gaps in Space Cybersecurity}
\begin{tabular}{|l|l|l|}
\hline
\textbf{Research Area} & \textbf{Current State} & \textbf{Identified Gaps} \\ \hline
Standardized Methodologies & Limited frameworks & Lack of lifecycle-integrated approaches \\ \hline
Quantitative Metrics & Early development & Insufficient validation and benchmarking \\ \hline
Tool Integration & Individual solutions & Lack of integrated assessment platforms \\ \hline
Cross-Domain Analysis & Isolated assessments & Limited multi-domain risk correlation \\ \hline
Validation Studies & Theoretical models & Insufficient real-world validation \\ \hline
\end{tabular}
\end{table}

The identified gaps in existing research and industry practices highlight the need for comprehensive, standardized risk assessment frameworks that can address the full lifecycle of space projects while providing practical, implementable solutions for industry adoption.

% ============================================================================
% CHAPTER 3: THEORETICAL FRAMEWORK
% ============================================================================
\chapter{Theoretical Framework}
\label{ch:framework}

This chapter presents the theoretical foundation of the standardized risk assessment framework developed in this research. The framework is designed to address the unique challenges and requirements associated with space systems, while simultaneously ensuring methodological consistency, repeatability, and adaptability across the entire lifecycle of space projects. By combining domain-specific considerations with a structured and scalable approach, the proposed framework aims to bridge the gap between traditional IT-centric risk models and the operational realities of the space sector.

\section{Framework Principles}

The proposed framework is built upon a set of core principles that ensure its applicability, flexibility, and robustness in the context of space missions:

\begin{itemize}
    \item \textbf{Lifecycle Integration}: Risk assessment methodologies that span from project conception to end-of-life
    \item \textbf{Standardization}: Consistent approaches enabling cross-project comparison and learning
    \item \textbf{Automation}: Tool-supported processes reducing manual effort and improving consistency
    \item \textbf{Scalability}: Methodologies applicable to projects of varying size and complexity
    \item \textbf{Traceability}: Clear documentation and reasoning for all risk decisions
\end{itemize}

\section{Risk Assessment Phases}

The framework identifies four distinct phases requiring specialized risk assessment approaches:
\begin{figure}
    \centering
    \includegraphics[width=\linewidth]{image.png}
    \caption{The Steps and Cycles in the risk management process}
    \label{fig:enter-label}
\end{figure}

\subsection{BID Phase Assessment}
The BID Phase Assessment represents the earliest opportunity to integrate cybersecurity considerations into a space project. Conducted during the proposal or pre-feasibility stage, this initial evaluation aims to align the project’s scope, budget, and planning assumptions with a realistic understanding of risk exposure. The assessment is based primarily on the project category (e.g., scientific mission, commercial payload, defense satellite) and its high-level functional and operational requirements. Although limited in technical detail at this stage, the process focuses on identifying fundamental cybersecurity constraints and potential mission-critical vulnerabilities that could affect project viability. This includes considerations such as mission criticality, reliance on third-party infrastructure (e.g., commercial ground stations), initial compliance with applicable standards (e.g., ECSS or ISO/IEC), and the presence of potential regulatory or geopolitical risks. The output of this phase informs the allocation of security resources, the initial definition of risk tolerance, and the inclusion of security planning in project proposals and contractual frameworks.

\subsection{Preliminary Risk Assessment}
The Preliminary Risk Assessment is conducted in the early stages of system design and requirement definition, typically during Phase 0/A of the mission lifecycle. Its objective is to create a structured foundation for subsequent security engineering and risk management activities. This phase involves the identification and categorization of key mission assets, such as onboard subsystems, data links, ground segment interfaces, and mission-critical software components. This phase promotes early visibility into systemic vulnerabilities, allowing for the definition of security requirements that can be integrated into system architecture, redundancy strategies, and operational planning. The outputs include a threat-asset matrix, initial risk register, and a prioritized list of areas requiring further analysis or mitigation in the next phase.

\subsection{Comprehensive Risk Assessment}
The Comprehensive Risk Assessment represents the most detailed and analytically intensive phase of the framework. Typically carried out during system development, integration, and testing phases (Phase B/C/D), it leverages a standardized taxonomy of threat types (e.g., nefarious activity, eavesdropping, physical sabotage), asset categories (e.g., space, ground, link, user), and lifecycle phases. The analysis employs both qualitative and quantitative methods, including risk scoring based on likelihood-impact matrices and attack path modeling. This phase also evaluates control effectiveness, environmental constraints, access complexity, and response capabilities. Special attention is paid to interdependencies among system components and cascading failure scenarios. The result is a comprehensive risk profile that supports informed decisions on design trade-offs, security control implementation, and mission assurance planning. Additionally, the assessment facilitates cross-domain alignment, particularly when projects involve international partners or dual-use technologies.

\subsection{Operational Risk Monitoring}
Operational Risk Monitoring begins at the launch phase and continues throughout in-orbit operations until the end of the system's active lifecycle. This phase shifts focus from design-time analysis to runtime threat detection and adaptive response. It incorporates continuous monitoring mechanisms, anomaly detection tools, telemetry analysis, and security event logging to assess real-time system health and potential security breaches. A distinguishing feature of this phase is the use of threat relationship analysis and attack path identification, which model how individual vulnerabilities or threat events could be linked across different system layers or mission phases to create multi-step attacks (e.g., ground-to-space intrusion vectors). This intelligence-driven approach enhances situational awareness and supports the formulation of dynamic countermeasures or contingency protocols. Operational risk monitoring also involves the periodic reevaluation of previously identified risks in light of new threat intelligence, system updates, or changes in mission configuration. The outputs are updated risk registers, incident response strategies, and data supporting long-term resilience assessments.

\section{Methodological Approach}

Each phase of the proposed risk assessment framework adopts tailored methodologies that address the unique needs and information availability of that particular stage in the project lifecycle. Despite the diversity of techniques employed, from high-level qualitative assessments during early conceptual phases to detailed quantitative modeling in later stages, methodological consistency is ensured through the use of shared principles, structured processes, and unified data representations.

At the core of this approach lies a standardized risk data model, which includes a common taxonomy of threats, asset classifications, evaluation criteria, and scoring metrics. This shared data structure facilitates the smooth transition of risk-related knowledge across lifecycle phases, reducing information loss and improving traceability. For example, threats identified during the Preliminary Risk Assessment are refined and expanded upon in the Comprehensive Risk Assessment using the same underlying classification system, enabling seamless aggregation, comparison, and revision of risk data over time.

Crucially, the framework is designed to be compatible with tool-supported implementation, facilitating partial automation of processes such as asset mapping, vulnerability identification, risk scoring, and documentation generation. This modular and data-centric approach ensures that each methodological component contributes coherently to an integrated and evolving risk profile throughout the entire mission lifecycle.

\section{Space Systems Risk Taxonomy}
The framework implements a dual-phase risk assessment methodology specifically designed for space systems cybersecurity. This approach combines preliminary rapid assessment capabilities with comprehensive detailed analysis, utilizing criteria-based evaluation and quadratic mean computation for robust risk quantification.

\subsection{Preliminary Risk Assessment}

The preliminary risk assessment provides rapid risk evaluation capabilities for initial threat identification and priority setting during early mission phases. This methodology focuses on threat-centric analysis with streamlined assessment criteria.

\subsubsection{Threat-Centric Assessment Approach}

The preliminary assessment employs a threat-centric methodology where each identified threat is evaluated against all relevant assets using five specialized criteria. This approach ensures comprehensive coverage while maintaining assessment efficiency:

\begin{itemize}
    \item \textbf{Vulnerability Level}: Evaluates known vulnerabilities and their mitigation status, from no known or resolved vulnerabilities to multiple unresolved critical vulnerabilities
    \item \textbf{Detection Probability}: Measures the likelihood that malicious activities will be detected, spanning continuous real-time monitoring with automated threat detection to complete absence of detection capabilities
    \item \textbf{Defense Capability}: Assesses comprehensive defense including effective mitigations, restricted access controls, and administrative privilege requirements, from multi-layered validated defenses to complete absence of relevant protections
    \item \textbf{Operational Impact}: Evaluates the effect on mission operations, from no impact thanks to redundancy with automated response to critical mission failure requiring extensive recovery
    \item \textbf{Recovery Time}: Measures time required to restore normal operations, from immediate restoration with automated procedures to inability to restore services
\end{itemize}

\subsubsection{Likelihood and Impact Computation}

The preliminary methodology employs a structured approach to derive likelihood and impact values from the five assessment criteria:

\textbf{Likelihood Calculation}: The first three criteria (Vulnerability Level, Detection Probability, and Defense Capability) contribute to likelihood assessment. Each criterion is scored on a 1-5 scale, and the overall likelihood is computed using quadratic mean to emphasize higher risk factors


\textbf{Impact Assessment}: The last two criteria (Operational Impact and Recovery Time) contribute to the impact score, offering focused evaluation of operational consequences and business continuity implications

$$V = \sqrt{\frac{1}{n}\sum_{i=1}^{n} C_i^2}$$

where $C_i$ represents the score for criterion $i$.


\textbf{Risk Level Determination}: Final risk levels are determined using the ISO 27005 risk matrix, combining computed likelihood and impact values to produce standardized risk classifications (Very Low, Low, Medium, High, Very High).

\subsection{Comprehensive Risk Assessment}

The comprehensive risk assessment provides detailed asset-focused analysis for thorough risk evaluation during detailed design and implementation phases. This methodology emphasizes asset vulnerabilities and multi-dimensional impact analysis.

\subsubsection{Asset-Centric Assessment Framework}

The comprehensive assessment employs an asset-centric approach where each asset is evaluated using nine specialized criteria covering both likelihood and impact dimensions. This methodology provides holistic risk understanding across technical and business domains:

\textbf{Likelihood Factors}:
\begin{itemize}
    \item \textbf{Dependency}: Evaluates asset criticality to mission operations, from non-critical support functions to essential mission-critical components
    \item \textbf{Penetration}: Assesses potential system access levels, from isolated user access to full privileged control of core infrastructure
    \item \textbf{Cyber Maturity}: Evaluates organizational cybersecurity governance, from mature audited systems to minimal cybersecurity procedures
    \item \textbf{Trust}: Assesses stakeholder trustworthiness, from strategic partners under strict control to unknown entities with unclear intent
\end{itemize}

\textbf{Impact Factors}:
\begin{itemize}
    \item \textbf{Performance}: Measures operational performance impact, from minimal effects to unacceptable degradation with no alternatives
    \item \textbf{Schedule}: Evaluates project timeline impact, from minimal delays to inability to achieve major milestones
    \item \textbf{Costs}: Assesses financial implications, from minimal cost increases to substantial budgetary impacts exceeding 15\%
    \item \textbf{Reputation}: Evaluates reputational consequences, from contained internal issues to irreparable international damage
    \item \textbf{Recovery}: Measures recovery time requirements, from limited damage with quick restoration to catastrophic long-term mission loss
\end{itemize}

\subsubsection{Multi-Dimensional Risk Calculation}

The comprehensive methodology employs sophisticated risk computation that integrates multiple assessment dimensions:

\textbf{Likelihood Computation}: The four likelihood criteria are combined using quadratic mean calculation to emphasize critical vulnerabilities:

\textbf{Impact Computation}: The five impact criteria are similarly combined using quadratic mean to capture the most severe potential consequences:

$$V_{comp} = \sqrt{\frac{1}{n}\sum_{k=1}^{n} I_k^2}$$

where $I_k$ represents the criterion scores.

\textbf{Integrated Risk Assessment}: The final risk level combines computed likelihood and impact using the same ISO 27005 risk matrix, ensuring consistency across assessment methodologies while providing enhanced granularity through multi-factor analysis.

\subsubsection{Quadratic Mean Rationale}

The framework employs quadratic mean (root mean square) rather than arithmetic mean for criterion aggregation based on risk assessment best practices:

\begin{itemize}
    \item \textbf{Emphasis on High-Risk Factors}: Quadratic mean assigns greater weight to higher individual scores, ensuring that critical vulnerabilities or severe impacts are not diluted by lower-scoring criteria
    \item \textbf{Mathematical Robustness}: The quadratic approach provides more stable and discriminating results compared to simple averaging, particularly important for security assessments where extreme values indicate critical risks
    \item \textbf{Conservative Risk Posture}: This approach supports conservative risk management by ensuring that high-severity individual factors drive overall risk assessments
    \item \textbf{Industry Alignment}: Quadratic mean aligns with established practices in safety-critical industries where maximum credible consequences drive risk evaluation
\end{itemize}


\section{Risk Treatment Strategies}
The framework incorporates a comprehensive catalog of cybersecurity controls derived from leading international standards and space-specific guidance documents. This multi-framework approach ensures coverage of diverse risk scenarios while maintaining alignment with established industry practices and regulatory requirements.

\subsection{Multi-Framework Control Integration}

The risk treatment strategy employs controls systematically extracted and harmonized from thirteen major cybersecurity frameworks and standards specifically relevant to space systems:

\begin{itemize}
    \item \textbf{ISO 27001}: International standard for information security management systems, providing baseline organizational security controls
    \item \textbf{NIST Cybersecurity Framework 2.0}: Comprehensive framework for managing cybersecurity risks across critical infrastructure
    \item \textbf{NIST IR 8270}: Space Systems Cybersecurity Guidelines providing space-specific security recommendations
    \item \textbf{NIST IR 8323 r1}: Foundational Cybersecurity Activities for IoT Device Manufacturers adapted for space systems
    \item \textbf{NIST IR 8401}: Cybersecurity Framework Profile for Hybrid Satellite Networks addressing space-terrestrial integration
    \item \textbf{NIST IR 8411}: Security Guidance for First Responder Mobile and Wearable Devices applicable to space operations
    \item \textbf{NIS2 Directive}: European cybersecurity requirements for critical infrastructure including space systems
    \item \textbf{NASA Best Practices Guide (BPG)}: NASA-specific cybersecurity practices for space missions
    \item \textbf{SPARTA}: Space cybersecurity framework addressing comprehensive space system security
    \item \textbf{BSI Profile for Space}: German Federal Office for Information Security space-specific guidance
    \item \textbf{BSI TR-03184}: Technical Guideline for secure satellite communication systems
    \item \textbf{METI Guidelines}: Japanese Ministry guidelines for space systems cybersecurity
    \item \textbf{UK Space Agency NCSC}: National Cyber Security Centre guidelines for space sector cybersecurity
\end{itemize}

\subsection{Control Categorization and Implementation}

The integrated control framework organizes 125 distinct cybersecurity controls into twelve strategic categories, each addressing specific aspects of space system security throughout the mission lifecycle:

\subsubsection{Policies \& Procedures}
Foundational governance controls establishing cybersecurity management frameworks, including information security policies, role definitions, resource allocation, and secure authentication procedures. These controls provide the organizational foundation for all subsequent security measures.

\subsubsection{Compliance}
Regulatory and legal compliance controls ensuring adherence to relevant cybersecurity requirements, intellectual property protection, independent security reviews, and assessment \& authorization procedures. These controls bridge organizational security practices with regulatory obligations.

\subsubsection{Risk Management}
Comprehensive risk assessment and management controls including threat modeling, criticality analysis, adaptive risk response mechanisms, third-party risk management, structured risk management processes, and business impact analysis. These controls provide systematic approaches to identifying, evaluating, and responding to cybersecurity risks.

\subsubsection{Security by Design}
Proactive security controls integrated into system design and development, including configuration management, secure coding standards, secure development lifecycle practices, cybersecurity-safe mode implementations, secure command modes, power system security, environment separation, and change management. These controls ensure security considerations are embedded throughout the system development process.

\subsubsection{Environmental \& Physical Security}
Physical protection controls addressing transport security, tamper protection, facility security, and environmental threat mitigation. These controls protect against physical attacks and environmental hazards that could compromise space system security.

\subsubsection{Network Security}
Communication and network protection controls including secure communication protocols, anti-counterfeit hardware measures, transmission security, physical port management, backdoor command elimination, resilient positioning and timing, smart contracts, alternative communication media, traffic flow security, network segmentation, cryptographic key management, message encryption, power masking, and RF encryption. These controls protect data in transit and secure communication pathways.

\subsubsection{Data Security}
Information protection controls covering data classification and labeling, comprehensive data management across all states, data loss prevention, backup procedures, information lifecycle management, data masking, real-time verification systems, process ID whitelisting, and tamper-resistant hardware. These controls ensure data confidentiality, integrity, and availability.

\subsubsection{Vulnerability Management}
Proactive security maintenance controls including malware protection, vulnerability management processes, software installation procedures, vulnerability scanning, security testing, software updates, protocol updates, software source control, trusted hardware development, and integrity checking mechanisms. These controls maintain system security posture against evolving threats.

\subsubsection{Access Management}
Identity and access control mechanisms including device authentication, access control policies, identity management, authentication information management, access rights management, authentication procedures, remote access management, multi-factor authentication, relay protection, session termination, insider threat protection, restricted zone access controls, and password security. These controls ensure only authorized entities can access system resources.

\subsubsection{Asset Management}
Comprehensive asset lifecycle controls including asset inventory maintenance, asset return procedures, equipment maintenance protocols, secure disposal processes, asset prioritization guidelines, and asset lifecycle management. These controls provide visibility and control over all system assets throughout their operational lifetime.

\subsubsection{Supply Chain Management}
Third-party risk controls addressing supplier security management, software version protection, software bill of materials generation, supply chain integrity assurance, cloud cybersecurity measures, and outsourced development oversight. These controls mitigate risks introduced through external dependencies and partnerships.

\subsubsection{Monitoring \& Alerting}
Continuous security monitoring controls including network and communications monitoring, intrusion detection and prevention systems, event detection communication, anomaly detection, cyber actor detection, telemetry monitoring, reinforcement learning systems, RF mapping, personnel monitoring, dependency confusion protection, and Security Information and Event Management (SIEM) integration. These controls provide real-time threat detection and response capabilities.

\subsubsection{Incident Response}
Security incident management controls including public relations management, incident response planning, incident threshold definition, emergency power systems, incident recovery procedures, secure cabling protocols, critical service delivery requirements, capacity management, and system redundancy. These controls ensure effective response to and recovery from security incidents.

\subsubsection{Capacity Building}
Organizational capability enhancement controls including information sharing mechanisms, cybersecurity awareness and training programs, and cyber threat intelligence collection and analysis. These controls strengthen overall organizational cybersecurity posture and situational awareness.

\subsubsection{Testing}
Security validation controls including software mission assurance, comprehensive software and hardware testing, dynamic code analysis, static code analysis, long duration testing, machine learning data integrity verification, dual authorization for on-orbit servicing, simulation testing, and detection process testing. These controls validate security measures and identify potential vulnerabilities.

\subsubsection{Continuous Improvement}
Ongoing enhancement controls including detection process improvement and oversight and governance mechanisms. These controls ensure the cybersecurity program evolves and improves based on operational experience and changing threat landscapes.

\subsubsection{Defense Capabilities}
Active protection controls including satellite maneuverability, defensive jamming and spoofing, deception and decoys, antenna nulling and adaptive filtering, physical seizure capabilities, sensor filtering and shuttering, defensive dazzling and blinding systems, and protective technology mechanisms. These controls provide active defense capabilities against sophisticated threats.

\subsection{Lifecycle-Aware Control Application}

The control framework recognizes that different cybersecurity measures are appropriate at different phases of the space mission lifecycle. Controls are mapped to specific lifecycle phases:

\begin{itemize}
    \item \textbf{All Phases}: Foundational controls applicable throughout the entire mission lifecycle
    \item \textbf{Phase B/C}: Design and development phase controls focusing on secure system architecture
    \item \textbf{Phase D}: Integration and testing phase controls ensuring secure implementation
    \item \textbf{Phase E}: Operations phase controls providing ongoing security during mission execution
    \item \textbf{Phase F}: End-of-life phase controls managing secure system decommissioning
\end{itemize}

\subsection{Segment-Specific Control Targeting}

Controls are strategically applied across different space system segments based on threat applicability and operational requirements:

\begin{itemize}
    \item \textbf{Ground Segment}: Controls protecting terrestrial infrastructure including ground stations, mission control centers, and data processing facilities
    \item \textbf{Space Segment}: Controls securing space-based assets including satellites, spacecraft, and orbital platforms
    \item \textbf{Link Segment}: Controls protecting communication pathways between space and ground segments
    \item \textbf{User Segment}: Controls securing end-user interfaces and data consumption points
    \item \textbf{Human Resources}: Controls addressing personnel security, training, and insider threat mitigation
\end{itemize}

\subsection{Risk-Based Control Prioritization}

The framework employs risk-based prioritization to guide control implementation decisions. Controls are evaluated based on their effectiveness in addressing specific risk criteria:

\begin{itemize}
    \item \textbf{Vulnerability Effectiveness}: Controls that reduce system vulnerability to exploitation
    \item \textbf{Mitigation Presence}: Controls that provide active threat mitigation capabilities
    \item \textbf{Detection Probability}: Controls that enhance threat detection and monitoring capabilities
    \item \textbf{Access Complexity}: Controls that increase the difficulty of unauthorized system access
    \item \textbf{Privilege Requirement}: Controls that enforce appropriate privilege levels for system operations
    \item \textbf{Response Delay}: Controls that improve incident response timing and effectiveness
    \item \textbf{Resilience Impact}: Controls that enhance system resilience and recovery capabilities
\end{itemize}

\subsection{Dynamic Control Integration and Assessment Synchronization}

The framework introduces an advanced methodology for dynamic control integration that maintains assessment consistency while enabling flexible security enhancement. This approach addresses the critical challenge of preserving risk evaluation integrity when cybersecurity controls are applied to existing threat assessments.

\subsubsection{Control-Assessment Integration Model}

The dynamic integration model employs a two-phase approach that separates baseline risk assessment from control-enhanced evaluation:

\begin{itemize}
    \item \textbf{Baseline Assessment Phase}: Initial threat and asset risk evaluation conducted without considering implemented controls, establishing fundamental risk posture
    \item \textbf{Control Enhancement Phase}: Application of selected cybersecurity controls with automatic recalculation of risk levels based on control effectiveness
    \item \textbf{Integrated Risk Profile}: Combined assessment showing both baseline and control-mitigated risk levels for comprehensive security planning
\end{itemize}

\subsubsection{Assessment Consistency Protection}

To maintain assessment integrity and prevent inconsistent risk evaluations, the framework implements assessment lock mechanisms:

\begin{itemize}
    \item \textbf{Read-Only Mode Activation}: Threat assessment interfaces become read-only once controls are applied, preventing manual modifications that could conflict with control-based adjustments
    \item \textbf{Baseline Preservation}: Original assessment values are preserved alongside control-modified values, enabling comparison and validation
    \item \textbf{Reset Capability}: Controls can be cleared to return to editable baseline assessment mode, allowing iterative refinement of risk evaluations
\end{itemize}

\subsubsection{Intelligent Control Compatibility}

The framework incorporates intelligent compatibility assessment that ensures controls are applied only where technically and operationally appropriate:

\begin{itemize}
    \item \textbf{Segment-Based Filtering}: Controls automatically apply only to compatible asset segments based on technical feasibility and operational requirements
    \item \textbf{Lifecycle Phase Alignment}: Control application respects mission phase constraints, ensuring controls are suggested only when implementable
    \item \textbf{Threat-Control Mapping}: Automated identification of control effectiveness against specific threats, optimizing security measure selection
\end{itemize}

\subsubsection{Real-Time Impact Analysis}

The framework provides real-time feedback on control selection effectiveness:

\begin{itemize}
    \item \textbf{Criteria Impact Visualization}: Immediate display of how selected controls affect specific risk assessment criteria
    \item \textbf{Threat Coverage Analysis}: Dynamic calculation of threat coverage levels based on control selection, identifying potential security gaps
    \item \textbf{Cumulative Effectiveness Assessment}: Real-time evaluation of multiple control interactions and their combined security impact
\end{itemize}

This comprehensive, multi-framework approach ensures that space system cybersecurity addresses the full spectrum of relevant threats while leveraging established best practices from both general cybersecurity and space-specific domains. The systematic categorization and lifecycle mapping of controls, combined with dynamic integration capabilities, provides practical guidance for implementing appropriate security measures throughout the space mission lifecycle while maintaining assessment consistency and enabling iterative security enhancement.

% ============================================================================
% CHAPTER 4: TOOL SUITE DESIGN - ENHANCED
% ============================================================================
\chapter{Tool Suite Design and Implementation}
\label{ch:implementation}

This chapter provides comprehensive details on the design philosophy, technical architecture, and implementation specifics of the Risk Assessment Tool Suite. The suite represents a practical embodiment of the theoretical framework, translating academic concepts into operational tools.

\section{Design Philosophy and Requirements}

\subsection{User-Centric Design Principles}

The tool suite development followed established human-computer interaction principles:

\begin{itemize}
    \item \textbf{Consistency}: Uniform interface elements across all tools
    \item \textbf{Accessibility}: Clear navigation and intuitive workflows
    \item \textbf{Efficiency}: Minimized cognitive load through logical grouping
    \item \textbf{Error Prevention}: Input validation and user guidance
    \item \textbf{Professional Appearance}: Enterprise-grade visual design
\end{itemize}

\subsection{Functional Requirements}

The tool suite addresses several critical functional requirements:

\begin{enumerate}
    \item \textbf{Multi-Phase Assessment Support}: Tools for different project lifecycle phases
    \item \textbf{Standardized Output}: Consistent reporting formats across tools
    \item \textbf{Data Interoperability}: Seamless data flow between components
    \item \textbf{Scalability}: Support for projects of varying complexity
    \item \textbf{Extensibility}: Architecture supporting future enhancements
\end{enumerate}

\section{System Architecture}

\subsection{Overall Architecture Design}

The Risk Assessment Tool Suite follows a modular monolithic architecture pattern, providing the benefits of modularity while maintaining simplicity for end users:

\subsection{Component Responsibilities}

Each component has clearly defined responsibilities:

\begin{itemize}
    \item \textbf{Main Interface}: Application launcher, tool coordination, output management
    \item \textbf{Individual Tools}: Specialized assessment functions, data collection, local processing
    \item \textbf{Shared Library}: Common functions, export capabilities, data validation
    \item \textbf{Output Manager}: File organization, report generation, format standardization
\end{itemize}

\section{Technical Implementation}

\subsection{Technology Stack Selection}

The choice of Python as the primary development language was driven by several factors:

\begin{table}[H]
\centering
\caption{Technology Selection Criteria}
\begin{tabular}{|l|l|l|}
\hline
\textbf{Aspect} & \textbf{Requirement} & \textbf{Python Advantage} \\ \hline
Rapid Development & Fast prototyping & Extensive libraries, readable syntax \\ \hline
GUI Development & Cross-platform UI & tkinter included, platform native \\ \hline
Data Processing & CSV, graph analysis & pandas, NetworkX ecosystem \\ \hline
Document Generation & Professional reports & python-docx, matplotlib integration \\ \hline
Deployment & Easy installation & pip package management \\ \hline
\end{tabular}
\end{table}

\subsection{Core Libraries and Dependencies}

The tool suite leverages several key Python libraries:

\begin{lstlisting}[language=Python, caption=Core Dependencies Structure]
# User Interface Framework
import tkinter as tk
from tkinter import ttk, messagebox, filedialog

# Data Processing and Analysis
import pandas as pd
import numpy as np
from datetime import datetime

# Visualization and Graphics
import matplotlib.pyplot as plt
import matplotlib.patches as patches
import networkx as nx
from PIL import Image, ImageTk, ImageDraw

# Document Generation
from docx import Document
from docx.shared import Inches
from docx.enum.text import WD_ALIGN_PARAGRAPH

# File and System Operations
import os
import sys
import subprocess
import csv
\end{lstlisting}

\subsection{Dynamic Control Management Implementation}

The tool suite implements advanced dynamic control management capabilities that seamlessly integrate cybersecurity controls with existing risk assessments while maintaining data integrity and assessment consistency.

\subsubsection{Control Database Architecture}

The control management system employs a structured CSV-based database containing 125 cybersecurity controls derived from 13 international frameworks:

\begin{lstlisting}[language=Python, caption=Control Database Structure]
# Control CSV Schema
CONTROL_COLUMNS = [
    'ID',                      # Unique control identifier
    'Control cluster',         # Control category grouping
    'Control title',          # Human-readable control name
    'Control',                # Detailed control description
    'Control description',    # Implementation guidance
    'Reference frameworks',   # Source frameworks (ISO27k, NIST, etc.)
    'Lifecycle phase',        # Applicable mission phases
    'Segment',               # Compatible asset segments
    'Threats addressed',      # Relevant threat categories
    'Criterio'               # Risk criteria affected
]

def load_controls_from_csv(self):
    """Load and structure cybersecurity controls from database"""
    try:
        with open('Control.csv', 'r', encoding='utf-8') as file:
            reader = csv.DictReader(file, delimiter=';')
            controls = {}
            
            for row in reader:
                if row['#'] and not row['#'].startswith('#'):
                    control_id = row['#']
                    controls[control_id] = {
                        'cluster': row['Control cluster'],
                        'title': row['Control title'],
                        'description': row['Control description'],
                        'reference': row['Reference frameworks'],
                        'lifecycle': row['Lifecycle phase'],
                        'segments': [s.strip() for s in row['Segment'].split(',')],
                        'threats': row['Threats addressed'],
                        'criteria': row['Criterio']
                    }
            return controls
    except Exception as e:
        messagebox.showerror("Error", f"Failed to load controls: {e}")
        return {}
\end{lstlisting}

\subsubsection{Real-Time Impact Analysis Engine}

The system provides real-time analysis of control selection impact through sophisticated calculation engines:

\begin{lstlisting}[language=Python, caption=Dynamic Impact Analysis Implementation]
def calculate_control_impact(self):
    """Calculate real-time impact of selected controls"""
    if not self.selected_controls:
        return self.clear_impact_display()
    
    # Initialize impact tracking
    criteria_impact = {}
    threat_coverage = {}
    control_clusters = {}
    
    for control_id in self.selected_controls:
        if control_id in self.controls:
            control = self.controls[control_id]
            
            # Track criteria improvements
            if control['criteria']:
                for criterion in control['criteria'].split(','):
                    criterion = criterion.strip()
                    if criterion:
                        criteria_impact.setdefault(criterion, 0)
                        criteria_impact[criterion] += 1
            
            # Track threat coverage
            if control['threats']:
                threats = [t.strip() for t in control['threats'].split(',')]
                for threat in threats:
                    if threat:
                        threat_coverage.setdefault(threat, 0)
                        threat_coverage[threat] += 1
            
            # Track control clustering
            cluster = control['cluster']
            control_clusters.setdefault(cluster, 0)
            control_clusters[cluster] += 1
    
    # Update display with calculated impacts
    self.update_impact_display(criteria_impact, threat_coverage, control_clusters)

def update_impact_display(self, criteria_impact, threat_coverage, control_clusters):
    """Update real-time impact visualization"""
    # Clear previous content
    for widget in self.impact_scrollable_frame.winfo_children():
        widget.destroy()
    
    # Display criteria impact analysis
    self.create_criteria_impact_section(criteria_impact)
    
    # Display threat coverage analysis
    self.create_threat_coverage_section(threat_coverage)
    
    # Display control cluster summary
    self.create_cluster_summary_section(control_clusters)
    
    # Refresh scrollable region
    self.impact_canvas.configure(scrollregion=self.impact_canvas.bbox("all"))
\end{lstlisting}

\subsubsection{Assessment Synchronization and Read-Only Mode}

The framework implements sophisticated assessment protection mechanisms to maintain data integrity:

\begin{lstlisting}[language=Python, caption=Assessment Protection Implementation]
def apply_controls_to_assessment(self):
    """Apply selected controls and enable read-only mode"""
    if not self.selected_controls:
        messagebox.showwarning("Warning", "No controls selected for application.")
        return
    
    try:
        # Calculate control impacts on threat criteria
        control_adjustments = self.calculate_criteria_adjustments()
        
        # Apply adjustments to threat assessment data
        for threat_id, adjustments in control_adjustments.items():
            if threat_id in self.threat_data:
                for asset, criteria_values in adjustments.items():
                    if asset in self.threat_data[threat_id]:
                        for criterion, adjustment in criteria_values.items():
                            current_value = self.threat_data[threat_id][asset][criterion]
                            # Apply improvement (reduce risk factors)
                            adjusted_value = max(1, current_value - adjustment)
                            self.threat_data[threat_id][asset][criterion] = adjusted_value
        
        # Recalculate risk levels with control impacts
        self.recalculate_risk_levels()
        
        # Enable read-only mode for threat assessment
        self.enable_read_only_mode()
        
        # Save applied controls configuration
        self.save_applied_controls()
        
        messagebox.showinfo("Success", 
                          f"Applied {len(self.selected_controls)} controls successfully.\n"
                          f"Threat Assessment is now in READ-ONLY mode.")
        
    except Exception as e:
        messagebox.showerror("Error", f"Failed to apply controls: {e}")

def enable_read_only_mode(self):
    """Enable read-only mode for threat assessment interface"""
    self.read_only_mode = True
    
    # Disable all threat assessment input widgets
    if hasattr(self, 'threat_window') and self.threat_window.winfo_exists():
        self.disable_threat_inputs(self.threat_window)
    
    # Update interface indicators
    self.update_read_only_indicators()

def clear_all_controls_and_enable_editing(self):
    """Clear controls and return to editable mode"""
    if self.read_only_mode:
        result = messagebox.askyesno(
            "Clear Controls", 
            "This will clear all applied controls and restore original assessment values.\n"
            "Threat Assessment will become editable again. Continue?"
        )
        if not result:
            return
    
    # Clear control applications
    self.selected_controls.clear()
    self.applied_controls.clear()
    
    # Restore original assessment values
    self.restore_original_assessment()
    
    # Disable read-only mode
    self.read_only_mode = False
    
    # Re-enable threat assessment inputs
    if hasattr(self, 'threat_window') and self.threat_window.winfo_exists():
        self.enable_threat_inputs(self.threat_window)
    
    messagebox.showinfo("Controls Cleared", 
                       "All controls cleared. Threat Assessment is now editable.")
\end{lstlisting}

\section{Scoring Methodology}

\subsection{Weighted Scoring Algorithm}

The tools implement a sophisticated weighted scoring system in the early phases based on cybersecurity best practices:

\begin{table}[H]
\centering
\caption{Cybersecurity Requirement Weights}
\begin{tabular}{|l|l|l|}
\hline
\textbf{Requirement Category} & \textbf{Weight} & \textbf{Justification} \\ \hline
Cybersecurity Requirements & 0.15 & Foundation of security posture \\ \hline
Security Architecture Constraints & 0.12 & Structural security limitations \\ \hline
Cryptographic Requirements & 0.10 & Data protection capabilities \\ \hline
Authentication \& Access Control & 0.11 & Identity management criticality \\ \hline
Supply Chain Security & 0.12 & Third-party risk exposure \\ \hline
Threat Modeling Guidelines & 0.08 & Proactive security planning \\ \hline
Security Compliance References & 0.09 & Regulatory alignment \\ \hline
Security Validation Requirements & 0.10 & Testing and verification \\ \hline
Incident Response Expectations & 0.07 & Operational preparedness \\ \hline
Data Protection and Privacy & 0.06 & Information safeguarding \\ \hline
\end{tabular}
\end{table}

\subsection{Risk Calculation Engine}

The risk calculation follows a multi-dimensional approach based on ISO 27005 methodology, implementing a two-stage process for accurate risk quantification:

\subsubsection{Stage 1: Likelihood and Impact Calculation}

The framework calculates likelihood and impact using quadratic mean (Root Mean Square) to properly weight extreme values:

\begin{lstlisting}[language=Python, caption=Likelihood and Impact Calculation]
def calculate_likelihood_impact(vulnerability, access_control, defense_capability):
    """
    Calculate likelihood using quadratic mean of relevant criteria
    """
    likelihood_factors = [vulnerability, access_control, defense_capability]
    
    # Quadratic mean calculation (RMS)
    likelihood = math.sqrt(sum(x**2 for x in likelihood_factors) / len(likelihood_factors))
    
    return likelihood

def calculate_impact(operational_impact, recovery_time):
    """
    Calculate impact using quadratic mean of consequence criteria
    """
    impact_factors = [operational_impact, recovery_time]
    
    # Quadratic mean calculation (RMS)
    impact = math.sqrt(sum(x**2 for x in impact_factors) / len(impact_factors))
    
    return impact
\end{lstlisting}

\subsubsection{Stage 2: ISO 27005 Risk Matrix Application}

After calculating likelihood and impact values, the framework applies the ISO 27005 risk assessment matrix to determine the final risk level:

\begin{table}[H]
\centering
\caption{ISO 27005 Risk Matrix}
\begin{tabular}{|c|c|c|c|c|c|}
\hline
\multirow{2}{*}{\textbf{Likelihood}} & \multicolumn{5}{c|}{\textbf{Impact}} \\ \cline{2-6}
& \textbf{1} & \textbf{2} & \textbf{3} & \textbf{4} & \textbf{5} \\ \hline
\textbf{5} & Medium & High & High & Very High & Very High \\ \hline
\textbf{4} & Low & Medium & High & High & Very High \\ \hline
\textbf{3} & Low & Low & Medium & High & High \\ \hline
\textbf{2} & Very Low & Low & Low & Medium & Medium \\ \hline
\textbf{1} & Very Low & Very Low & Low & Low & Low \\ \hline
\end{tabular}
\end{table}

\begin{lstlisting}[language=Python, caption=Complete Risk Calculation Implementation]
import math

def calculate_comprehensive_risk(vulnerability, access_control, defense_capability, 
                               operational_impact, recovery_time):
    """
    Complete risk calculation using quadratic mean and ISO 27005 matrix
    """
    # Stage 1: Calculate likelihood and impact using quadratic mean
    likelihood_raw = calculate_likelihood_impact(vulnerability, access_control, defense_capability)
    impact_raw = calculate_impact(operational_impact, recovery_time)
    
    # Normalize to 0-1 scale
    likelihood = (likelihood_raw - 1) / (5 - 1)
    impact = (impact_raw - 1) / (5 - 1)
    
    # Stage 2: Apply ISO 27005 risk matrix
    risk_matrix = {
        ("Very High", "Very High"): "Very High", ("Very High", "High"): "Very High",
        ("Very High", "Medium"): "High", ("Very High", "Low"): "High",
        ("Very High", "Very Low"): "Medium", ("High", "Very High"): "Very High",
        ("High", "High"): "High", ("High", "Medium"): "High",
        ("High", "Low"): "Medium", ("High", "Very Low"): "Low",
        ("Medium", "Very High"): "High", ("Medium", "High"): "High",
        ("Medium", "Medium"): "Medium", ("Medium", "Low"): "Low",
        ("Medium", "Very Low"): "Low", ("Low", "Very High"): "Medium",
        ("Low", "High"): "Medium", ("Low", "Medium"): "Low",
        ("Low", "Low"): "Low", ("Low", "Very Low"): "Very Low",
        ("Very Low", "Very High"): "Low", ("Very Low", "High"): "Low",
        ("Very Low", "Medium"): "Low", ("Very Low", "Low"): "Very Low",
        ("Very Low", "Very Low"): "Very Low"
    }
    
    risk_level = risk_matrix.get(likelihood, impact)
    
    return {
        'likelihood_raw': likelihood_raw,
        'impact_raw': impact_raw,
        'likelihood': likelihood,
        'impact': impact,
        'risk_level': risk_level
    }
\end{lstlisting}

\section{User Interface Design}

\subsection{Interface Design Patterns}

The tool suite employs consistent design patterns across all components:

\begin{itemize}
    \item \textbf{Header Section}: Logo, title, and navigation elements
    \item \textbf{Content Area}: Input forms, data displays, and controls
    \item \textbf{Action Bar}: Primary action buttons (Assess, Export, Clear)
    \item \textbf{Status Bar}: Progress indicators and status messages
    \item \textbf{Footer}: Application information and credits
\end{itemize}

\subsection{Visual Design Enhancements}

The interface incorporates professional visual elements with enhanced logo integration and branding:

\begin{lstlisting}[language=Python, caption=Logo Integration and Visual Enhancement]
# Logo implementation with rounded corners
def create_rounded_image(self, image, radius=20):
    """Create rounded corners for logo display"""
    size = image.size
    mask = Image.new('L', size, 0)
    draw = ImageDraw.Draw(mask)
    draw.rounded_rectangle([0, 0] + list(size), 
                          radius=radius, fill=255)
    
    rounded_image = Image.new('RGBA', size, (0, 0, 0, 0))
    rounded_image.paste(image, (0, 0))
    rounded_image.putalpha(mask)
    
    return rounded_image

# Logo loading and scaling implementation
def load_and_scale_logo(self, logo_path, target_size=(80, 80)):
    """Load and scale logo with error handling"""
    try:
        logo = Image.open(logo_path)
        logo = logo.resize(target_size, Image.Resampling.LANCZOS)
        logo_rounded = self.create_rounded_image(logo)
        return ImageTk.PhotoImage(logo_rounded)
    except FileNotFoundError:
        self.show_message("Logo file not found")
        return None
\end{lstlisting}

\subsection{Enhanced Color Scheme and Styling}

The interface utilizes a professional color scheme with comprehensive styling support:

\begin{lstlisting}[language=Python, caption=Enhanced UI Styling Configuration]
# Color scheme for professional appearance
COLORS = {
    'primary': '#2E86AB',      # Professional blue
    'secondary': '#A23B72',    # Accent color
    'success': '#F18F01',      # Warning/attention
    'background': '#F5F5F5',   # Light gray background
    'dark': '#333333',         # Dark gray text
    'border': '#CCCCCC',       # Light border
    'light': '#FFFFFF',        # White background
    'error': '#DC3545',        # Error red
    'warning': '#FFC107'       # Warning yellow
}

# Font configuration with hierarchical typography
FONTS = {
    'title': ('Segoe UI', 16, 'bold'),
    'heading': ('Segoe UI', 12, 'bold'),
    'body': ('Segoe UI', 10),
    'small': ('Segoe UI', 8),
    'help_title': ('Segoe UI', 14, 'bold'),
    'help_section': ('Segoe UI', 11, 'bold'),
    'help_body': ('Segoe UI', 9)
}

# Widget styling with enhanced accessibility
WIDGET_STYLES = {
    'button': {
        'font': FONTS['body'],
        'bg': COLORS['primary'],
        'fg': 'white',
        'relief': 'flat',
        'padx': 20,
        'pady': 8,
        'cursor': 'hand2'
    },
    'frame': {
        'bg': COLORS['background'],
        'relief': 'solid',
        'bd': 1
    },
    'help_text': {
        'bg': COLORS['light'],
        'fg': COLORS['dark'],
        'wrap': 'word',
        'padx': 15,
        'pady': 10
    }
}
\end{lstlisting}

\subsection{Comprehensive Help System Architecture}

A sophisticated help system has been implemented across all tools to enhance user experience and provide comprehensive guidance. The help system includes specialized documentation for advanced features such as the dynamic Controls Management system with real-time impact analysis:

\begin{lstlisting}[language=Python, caption=Enhanced Help System Implementation]
def show_help(self):
    """Display comprehensive help window with structured content"""
    help_window = tk.Toplevel(self.root)
    help_window.title("Risk Assessment Tool - User Guide")
    help_window.geometry("1200x700")  # Enhanced size for Controls Management help
    help_window.configure(bg=COLORS['background'])
    help_window.resizable(True, True)
    
    # Create main container with scrollbar
    main_frame = tk.Frame(help_window, bg=COLORS['background'])
    main_frame.pack(fill='both', expand=True, padx=10, pady=10)
    
    # Scrollable canvas for complex help content
    canvas = tk.Canvas(main_frame, bg=COLORS['background'], highlightthickness=0)
    scrollbar = tk.Scrollbar(main_frame, orient="vertical", command=canvas.yview)
    scrollable_frame = tk.Frame(canvas, bg=COLORS['background'])
    
    scrollable_frame.bind(
        "<Configure>",
        lambda e: canvas.configure(scrollregion=canvas.bbox("all"))
    )
    
    canvas.create_window((0, 0), window=scrollable_frame, anchor="nw")
    canvas.configure(yscrollcommand=scrollbar.set)
    
    # Enhanced help content including Controls Management
    self.create_help_content_sections(scrollable_frame)
    
    canvas.pack(side="left", fill="both", expand=True, padx=(0, 5))
    scrollbar.pack(side="right", fill="y")
    
    # Setup mouse wheel scrolling
    self.setup_global_mousewheel(scrollable_frame, canvas)

def create_help_content_sections(self, parent):
    """Create structured help sections including Controls Management"""
    help_sections = [
        ("Tool Overview", "Comprehensive cybersecurity risk assessment capabilities"),
        ("Assessment Workflows", "Threat-centric and asset-centric analysis modes"),
        ("Controls Management", "Dynamic cybersecurity controls integration with real-time impact analysis"),
        ("Read-Only Mode", "Assessment protection and control integration consistency"),
        ("Export and Reporting", "Professional documentation generation"),
        ("Advanced Features", "Enhanced UI, scrolling protection, and workflow optimization")
    ]
    
    for title, description in help_sections:
        self.create_help_section(parent, title, description)
- Review results in the output area
- Export findings using the export functionality

3. ASSESSMENT METHODOLOGY
The tool implements a structured risk assessment approach:
- Asset identification and categorization
- Threat modeling and vulnerability analysis
- Risk calculation using weighted scoring algorithms
- Mitigation strategy recommendations
- Compliance mapping to relevant standards

4. DATA MANAGEMENT
- Import/Export: Use CSV and JSON formats for data exchange
- Legacy Integration: Import data from previous assessments
- Output Management: Results are saved in standardized formats
- Backup: Regular data backup recommended

5. ADVANCED FEATURES
- Multi-criteria decision analysis support
- Scenario-based threat modeling
- Automated report generation
- Integration with attack graph analysis
- Real-time risk monitoring capabilities

6. TROUBLESHOOTING
- Verify input data format and completeness
- Check file permissions for import/export operations
- Ensure all required fields are completed
- Contact support for technical issues

7. BEST PRACTICES
- Regular assessment updates throughout project lifecycle
- Collaborative review with cybersecurity experts
- Documentation of assessment assumptions
- Integration with project risk management processes
- Continuous monitoring of emerging threats

"""
    return content
\end{lstlisting}

\subsection{Optimized Button Layout and User Experience}

The tool suite features optimized button layouts designed for efficient workflow and intuitive navigation:

\begin{lstlisting}[language=Python, caption=Enhanced Button Layout Implementation]
def create_optimized_button_layout(self):
    """Create organized button layout with logical grouping"""
    # Main action area with centered primary functions
    action_frame = tk.Frame(self.root, bg=COLORS['background'])
    action_frame.pack(pady=20)
    
    # Left section: Import/Export operations
    left_frame = tk.Frame(action_frame, bg=COLORS['background'])
    left_frame.pack(side='left', padx=20)
    
    tk.Button(left_frame, text="Import Data", 
             command=self.import_data,
             **WIDGET_STYLES['button']).pack(pady=5)
    
    tk.Button(left_frame, text="Export Results", 
             command=self.export_results,
             **WIDGET_STYLES['button']).pack(pady=5)
    
    # Center section: Primary assessment function
    center_frame = tk.Frame(action_frame, bg=COLORS['background'])
    center_frame.pack(side='left', padx=40)
    
    # Prominent main action button
    main_button = tk.Button(center_frame, text="START ASSESSMENT",
                           command=self.start_assessment,
                           font=FONTS['heading'],
                           bg=COLORS['success'],
                           fg='white',
                           padx=30, pady=15,
                           relief='raised',
                           cursor='hand2')
    main_button.pack()
    
    # Right section: Support and legacy functions
    right_frame = tk.Frame(action_frame, bg=COLORS['background'])
    right_frame.pack(side='right', padx=20)
    
    tk.Button(right_frame, text="Legacy Data", 
             command=self.import_legacy,
             **WIDGET_STYLES['button']).pack(pady=5)
    
    tk.Button(right_frame, text="Help", 
             command=self.show_help,
             **WIDGET_STYLES['button']).pack(pady=5)
\end{lstlisting}

\section{Data Management Architecture}

\subsection{Data Flow Design}

The tool suite implements a structured data flow pattern:

\begin{enumerate}
    \item \textbf{Input Validation}: User input sanitization and format checking
    \item \textbf{Data Processing}: Transformation and calculation procedures
    \item \textbf{Intermediate Storage}: Temporary data structures for processing
    \item \textbf{Output Generation}: Formatted results and report creation
    \item \textbf{Export Management}: File writing and organization
\end{enumerate}

\subsection{File Management System}

The output management system ensures organized and consistent file handling:

\begin{lstlisting}[language=Python, caption=Output Management Implementation]
class OutputManager:
    def __init__(self, base_path="Output"):
        self.base_path = base_path
        self.ensure_output_directory()
    
    def ensure_output_directory(self):
        """Create output directory if it doesn't exist"""
        if not os.path.exists(self.base_path):
            os.makedirs(self.base_path)
    
    def generate_filename(self, tool_name, file_type, timestamp=True):
        """Generate standardized filename"""
        if timestamp:
            time_str = datetime.now().strftime("%Y%m%d_%H%M%S")
            return f"{tool_name}_{time_str}.{file_type}"
        else:
            return f"{tool_name}.{file_type}"
    
    def get_output_path(self, filename):
        """Get full path for output file"""
        return os.path.join(self.base_path, filename)
\end{lstlisting}

\section{Performance Optimization}

\subsection{Computational Efficiency}

The tools implement several optimization strategies:

\begin{itemize}
    \item \textbf{Lazy Loading}: UI components loaded on demand
    \item \textbf{Caching}: Frequent calculations cached for reuse
    \item \textbf{Vectorized Operations}: NumPy arrays for numerical computations
    \item \textbf{Memory Management}: Efficient data structure usage
\end{itemize}

\subsection{User Experience Optimization}

Performance considerations for user interaction:

\begin{lstlisting}[language=Python, caption=Performance Optimization Example]
def optimized_risk_calculation(self, data_matrix):
    """Vectorized risk calculation for large datasets"""
    # Convert to NumPy arrays for efficient computation
    data_array = np.array(data_matrix)
    weights_array = np.array(list(self.weights.values()))
    
    # Vectorized calculation
    risk_scores = np.dot(data_array, weights_array)
    
    # Batch categorization
    risk_levels = np.select(
        [risk_scores <= 0.2, risk_scores <= 0.4, 
         risk_scores <= 0.6, risk_scores <= 0.8],
        ['Very Low', 'Low', 'Medium', 'High'],
        default='Very High'
    )
    
    return risk_scores, risk_levels
\end{lstlisting}

\section{Integration Capabilities}

\subsection{API Design Considerations}

Although the current implementation focuses on standalone operation, the architecture supports future API development:

\begin{itemize}
    \item \textbf{Modular Functions}: Core assessment logic separated from UI
    \item \textbf{Standardized Data Formats}: CSV-compatible data structures
    \item \textbf{Error Handling}: Comprehensive exception management
    \item \textbf{Documentation}: Function-level documentation for API exposure
\end{itemize}

\subsection{External System Integration}

The tool suite design accommodates integration with external systems:

\begin{itemize}
    \item \textbf{Data Import}: CSV data import capabilities
    \item \textbf{Export Formats}: Multiple output formats for different systems
    \item \textbf{Configuration Files}: Externally configurable parameters
    \item \textbf{Plugin Architecture}: Extensible component structure
\end{itemize}

\subsection{Cross-Tool Compatibility and Criteria Mapping}

The framework implements intelligent criteria mapping between different assessment phases to ensure seamless data flow and consistency:

\subsubsection{Phase 0/A to Comprehensive Assessment Mapping}

The preliminary assessment tool (Phase 0/A) uses a streamlined five-criteria framework that maps effectively to the comprehensive seven-criteria framework:

\begin{itemize}
    \item \textbf{Detection Probability} (Phase 0/A) → \textbf{Detection} (Criterion 2, Comprehensive)
    \item \textbf{Defense Capability} (Phase 0/A) → Maps to multiple comprehensive criteria:
    \begin{itemize}
        \item \textbf{Mitigation} (Criterion 1) - Countermeasures effectiveness
        \item \textbf{Access} (Criterion 3) - Access control complexity  
        \item \textbf{Privilege} (Criterion 4) - Required privilege levels
    \end{itemize}
\end{itemize}

This mapping approach ensures that:
\begin{itemize}
    \item Data from preliminary assessments can be imported into comprehensive tools
    \item Assessment continuity is maintained across project phases
    \item Higher resolution analysis builds upon earlier findings
    \item No critical security information is lost during phase transitions
\end{itemize}

The Defense Capability criterion in Phase 0/A encompasses comprehensive defense aspects including mitigations, access controls, and privilege requirements, reflecting the multi-dimensional nature of cybersecurity defense systems.

\section{Technical Challenges and Solutions}

\subsection{Asset Data Management Enhancement}

\subsubsection{Centralized Asset Data Loading}

The tools were enhanced to implement centralized asset data management through CSV-based loading:

\begin{lstlisting}[language=Python, caption=Enhanced Asset Loading Implementation]
def load_asset_categories_from_csv(self):
    """Load asset categories from Asset.csv with error handling"""
    assets_file = os.path.join(get_base_path(), "Asset.csv")
    asset_categories = []
    seen_combinations = set()
    
    try:
        with open(assets_file, 'r', encoding='utf-8') as csvfile:
            reader = csv.DictReader(csvfile, delimiter=';')
            for row in reader:
                category = row.get('categories', '').strip()
                subcategory = row.get('subCategories', '').strip()
                
                combination = (category, subcategory)
                if combination not in seen_combinations and category and subcategory:
                    seen_combinations.add(combination)
                    asset_categories.append(combination)
        
        return asset_categories
        
    except FileNotFoundError:
        # Graceful fallback to hardcoded assets
        return self.get_default_asset_categories()
    except Exception as e:
        self.log_error(f"Error loading assets: {e}")
        return self.get_default_asset_categories()
\end{lstlisting}

\subsubsection{Data Consistency and Validation}

The implementation includes robust data validation and consistency checking:

\begin{itemize}
    \item \textbf{Delimiter Handling}: Proper CSV parsing with semicolon delimiters
    \item \textbf{Duplicate Prevention}: Automatic removal of duplicate asset combinations
    \item \textbf{Error Recovery}: Graceful fallback to default assets when CSV loading fails
    \item \textbf{Data Integrity}: Validation of required fields and data completeness
\end{itemize}


\subsection{Error Handling and Robustness}

\subsubsection{Comprehensive Error Management}

A robust error handling framework was implemented to ensure operational reliability:

\begin{lstlisting}[language=Python, caption=Enhanced Error Handling Framework]
class ErrorHandler:
    """Centralized error handling with graceful degradation"""
    
    def handle_color_reference_error(self, color_key, fallback='#CCCCCC'):
        """Handle missing color references gracefully"""
        try:
            return COLORS[color_key]
        except KeyError:
            self.log_warning(f"Color '{color_key}' not found, using fallback")
            return fallback
    
    def handle_file_operation_error(self, operation, filepath, exception):
        """Handle file operation errors with user feedback"""
        error_message = f"File operation '{operation}' failed for {filepath}: {exception}"
        self.log_error(error_message)
        self.show_user_notification(error_message, severity='error')
        return None
    
    def ensure_graceful_degradation(self, primary_function, fallback_function):
        """Implement graceful degradation for critical functions"""
        try:
            return primary_function()
        except Exception as e:
            self.log_error(f"Primary function failed: {e}")
            return fallback_function()
\end{lstlisting}

\subsubsection{User Experience Continuity}

The error handling system prioritizes user experience continuity:

\begin{itemize}
    \item \textbf{Graceful Fallbacks}: Automatic fallback to safe alternatives when errors occur
    \item \textbf{User Notification}: Clear, non-technical error messages for end users
    \item \textbf{Operation Continuity}: Tools remain functional even when non-critical components fail
    \item \textbf{Recovery Procedures}: Automated recovery mechanisms for common failure scenarios
\end{itemize}

\subsection{Asset Data Centralization and Management}

\subsubsection{Transition from Static to Dynamic Asset Loading}

A significant enhancement implemented during development was the transition from static asset definitions to dynamic CSV-based asset loading. This improvement addressed several critical limitations:

\begin{itemize}
    \item \textbf{Maintainability}: Centralized asset data eliminates code duplication across tools
    \item \textbf{Flexibility}: Easy addition or modification of asset categories without code changes
    \item \textbf{Consistency}: Unified asset taxonomy across all assessment phases
    \item \textbf{Scalability}: Support for larger and more complex asset hierarchies
\end{itemize}

\subsubsection{Asset.csv Structure and Implementation}

The centralized asset database follows a structured format designed for comprehensive coverage of space system components:

\begin{table}[H]
\centering
\caption{Asset.csv Structure and Content Distribution}
\begin{tabular}{|l|l|l|}
\hline
\textbf{Category} & \textbf{Subcategories} & \textbf{Asset Count} \\ \hline
Ground & Ground Stations, Mission Control, Data Processing & 12 \\ \hline
Space & Platform, Payload & 8 \\ \hline
Link & Communication Links & 6 \\ \hline
User & User Segments & 8 \\ \hline
\textbf{Total} & \textbf{9 Unique Combinations} & \textbf{34 Assets} \\ \hline
\end{tabular}
\end{table}

The implementation ensures robust data loading with comprehensive error handling:

\begin{lstlisting}[language=Python, caption=Robust Asset Loading with Validation]
def load_assets_from_csv(self):
    """Enhanced asset loading with comprehensive validation"""
    assets_file = os.path.join(get_base_path(), "Asset.csv")
    
    try:
        with open(assets_file, 'r', encoding='utf-8') as csvfile:
            reader = csv.DictReader(csvfile, delimiter=';')
            
            # Validate CSV structure
            required_fields = ['categories', 'subCategories', 'asset']
            if not all(field in reader.fieldnames for field in required_fields):
                raise ValueError("Missing required CSV fields")
            
            assets = []
            for row_num, row in enumerate(reader, start=2):
                # Validate row completeness
                if all(row.get(field, '').strip() for field in required_fields):
                    assets.append({
                        'category': row['categories'].strip(),
                        'subcategory': row['subCategories'].strip(),
                        'asset': row['asset'].strip()
                    })
                else:
                    self.log_warning(f"Incomplete data in row {row_num}")
            
            if not assets:
                raise ValueError("No valid assets found in CSV")
                
            return assets
            
    except Exception as e:
        self.log_error(f"Asset loading failed: {e}")
        return self.get_default_assets()
\end{lstlisting}

\subsection{Path Resolution and Executable Environment Handling}

\subsubsection{Cross-Environment Path Management}

A critical technical challenge addressed was ensuring proper file path resolution across different execution environments (development vs. executable):

\begin{lstlisting}[language=Python, caption=Robust Path Resolution Implementation]
def get_base_path():
    """Enhanced path resolution for multiple execution contexts"""
    if getattr(sys, 'frozen', False):
        # Running as compiled executable
        if hasattr(sys, '_MEIPASS'):
            # PyInstaller temporary folder
            base_path = sys._MEIPASS
        else:
            # Standard executable directory
            base_path = os.path.dirname(sys.executable)
    else:
        # Running as Python script
        base_path = os.path.dirname(os.path.abspath(__file__))
    
    # Validate path exists and log for debugging
    if not os.path.exists(base_path):
        raise FileNotFoundError(f"Base path not found: {base_path}")
    
    return base_path
\end{lstlisting}

\subsubsection{Data File Inclusion Strategy}

The PyInstaller configuration was enhanced to ensure proper inclusion of data files in executables:

\begin{itemize}
    \item \textbf{Asset Data}: `Asset.csv` included for dynamic asset loading
    \item \textbf{Visual Resources}: Logo files and icons properly embedded
    \item \textbf{Configuration Files}: Threat relationship data and export templates
    \item \textbf{Shared Libraries}: Common utility functions packaged correctly
\end{itemize}

\subsection{Quality Assurance and Testing Improvements}

\subsubsection{Systematic Testing Framework}

The development process incorporated systematic testing to identify and resolve issues:

\begin{enumerate}
    \item \textbf{Unit Testing}: Individual function validation with edge case handling
    \item \textbf{Integration Testing}: Cross-tool data flow and compatibility verification
    \item \textbf{Executable Testing}: Comprehensive testing of compiled versions
    \item \textbf{Unicode Compatibility}: Systematic scanning for encoding issues
    \item \textbf{Path Resolution}: Testing across development and deployment environments
\end{enumerate}

\subsubsection{Continuous Improvement Process}

The implementation follows a continuous improvement methodology:

\begin{itemize}
    \item \textbf{Issue Identification}: Systematic identification of runtime issues
    \item \textbf{Root Cause Analysis}: Thorough investigation of underlying causes
    \item \textbf{Solution Implementation}: Targeted fixes with minimal impact
    \item \textbf{Regression Testing}: Validation that fixes don't introduce new issues
    \item \textbf{Documentation Updates}: Maintenance of accurate technical documentation
\end{itemize}

% ============================================================================
% CHAPTER 5: TOOL SUITE COMPONENTS
% ============================================================================
\chapter{Risk Assessment Tool Suite Architecture}
\label{ch:components}

This chapter provides a comprehensive technical analysis of the integrated Risk Assessment Tool Suite developed for space mission cybersecurity. The suite follows a phase-appropriate methodology aligned with ECSS and ISO 27005 standards, implementing quantitative risk assessment through specialized modules for each project lifecycle stage.

\section{BID Phase Assessment Module}
\label{sec:bid_tool}

\subsection{Technical Implementation}

The BID Phase module (0-BID.exe) implements a weighted scoring system for Invitation to Tender (ITT) evaluation through three core algorithmic components:

\begin{enumerate}
    \item \textbf{Score Matrix}: 11 cybersecurity criteria with 4-level scoring (1=Low to 4=High risk)
    \item \textbf{Risk Calculation Engine}: 
    \begin{equation}
        R = \sum_{i=1}^{11} \frac{(V_i - 1) \times W_i}{3}
    \end{equation}
    where $V_i$ is criterion value and $W_i$ is dynamic weight
    \item \textbf{Inapplicability Handler}: Redistributes weights when criteria are marked non-applicable
\end{enumerate}

The tool's architecture features:

\begin{itemize}
    \item Dynamic CSV/Word report generation with timestamped filenames
    \item Multi-threaded execution for responsive UI during calculations
    \item High-DPI aware Tkinter interface with automatic scaling
\end{itemize}

\subsection{Assessment Methodology}

The module implements a four-stage assessment workflow:

\begin{lstlisting}[language=Python, caption=Core Assessment Algorithm]
def calculate_risk(self):
    """Implements ECSS-M-ST-80C compliant risk scoring"""
    total_score = 0.0
    for criterion in self.criteria:
        value = self.get_criterion_value(criterion)
        weight = self.get_dynamic_weight(criterion)
        if not self.is_applicable(criterion):
            weight = 0
        normalized = (value - 1) * weight / 3  # Scale to 0-1 range
        total_score += normalized
    return min(1.0, total_score)  # Cap at 1.0
\end{lstlisting}

Key assessment parameters:

\begin{table}[h]
\centering
\caption{BID Module Risk Classification}
\begin{tabular}{ll}
\hline
Score Range & Risk Level \\ \hline
0-0.1 & Very Low \\
0.1-0.4 & Low \\
0.4-0.7 & Medium \\
0.7-0.9 & High \\
0.9-1.0 & Very High \\ \hline
\end{tabular}
\end{table}

\subsection{Output Generation}

The tool produces two report formats through a unified export handler:

\begin{lstlisting}[language=Python, caption=Report Generation Logic]
def generate_reports(self):
    """Produces both CSV and Word outputs"""
    timestamp = datetime.now().strftime("%Y%m%d_%H%M%S")
    if DOCX_AVAILABLE:
        self._generate_word_report(f"BID_Assessment_{timestamp}.docx")
    self._generate_csv_report(f"BID_Risk_Assessment_{timestamp}.csv")
\end{lstlisting}

Report contents include:
\begin{itemize}
    \item Complete score matrix with justification texts
    \item Weight redistribution audit trail
    \item Risk level visualization with color coding
\end{itemize}

\section{Preliminary Risk Assessment Module}
\label{sec:phase0a_tool}

\subsection{Technical Architecture}

The Phase 0-A tool (1-Risk\_Assessment\_0-A.exe) implements:

\begin{itemize}
    \item Asset-threat matrix with 11 threat types and dynamic asset loading
    \item Quadratic mean calculation:
    \item ISO 27005 risk matrix implementation
\end{itemize}

Key technical features:

\begin{lstlisting}[language=Python, caption=Risk Calculation Core]
def calculate_risk(self, likelihood, impact):
    """Implements ISO 27005 risk matrix"""
    matrix = {
        ('Very High', 'Very High'): 'Very High',
        ('High', 'High'): 'High',
        # Full matrix implementation
    }
    return matrix.get((likelihood, impact), 'Undefined')
\end{lstlisting}

\subsection{Data Model}

The tool employs a hierarchical data structure:

Asset categories are loaded dynamically from Asset.csv with fallback to default values:

\begin{lstlisting}[language=Python, caption=Asset Loading Logic]
def load_assets(self):
    """Loads asset categories with CSV fallback"""
    try:
        with open('Asset.csv', 'r', encoding='utf-8') as f:
            reader = csv.DictReader(f, delimiter=';')
            return {(row['categories'], row['subCategories']) 
                   for row in reader}
    except FileNotFoundError:
        return DEFAULT_ASSETS  # Predefined tuple set
\end{lstlisting}

\subsection{Reporting System}

The module features comprehensive Word reporting with:

\begin{itemize}
    \item Threat-specific countermeasure listings
    \item Asset risk heatmaps
    \item Criteria reference tables
    \item Automatic diagram generation
\end{itemize}

\section{Comprehensive Risk Assessment Tool}

\subsection{Purpose and Scope}

The Comprehensive Risk Assessment Tool is designed for detailed cybersecurity evaluation during the design and implementation phases of space missions. It provides a systematic approach to assess risks at the component level, integrating both threat-centric and asset-centric analysis methodologies. The tool addresses the critical need for rigorous risk assessment as system designs mature and implementation details become finalized.

\subsection{Methodology}

The tool implements a dual-path assessment methodology combining:

\begin{enumerate}
\item \textbf{Threat Analysis}: Evaluation of specific threats against all mission assets using 7 criteria:
\begin{itemize}
\item Vulnerability effectiveness
\item Mitigation presence
\item Detection probability
\item Access complexity
\item Privilege requirement
\item Response delay
\item Resilience impact
\end{itemize}

\item \textbf{Asset Analysis}: Comprehensive evaluation of each asset using 9 criteria:
\begin{itemize}
    \item Dependency
    \item Penetration
    \item Cyber maturity
    \item Trust
    \item Performance
    \item Schedule
    \item Costs
    \item Reputation
    \item Recovery
\end{itemize}

\item \textbf{Risk Calculation}: Combined risk evaluation using:
\begin{itemize}
    \item Quadratic mean for criteria aggregation
    \item ISO 27005 risk matrix for likelihood-impact combination
    \item Normalized scoring (1-5) converted to risk categories
\end{itemize}
\end{enumerate}

\subsection{Key Features}

The tool's architecture implements several innovative features:

\begin{itemize}
\item \textbf{Dual Assessment Framework}:
\begin{itemize}
\item Threat window analyzes each threat across all assets
\item Asset window evaluates each asset against all criteria
\item Automatic cross-referencing of results
\end{itemize}

\item \textbf{Advanced Calculation Engine}:
\begin{itemize}
    \item Context-aware calculations (threat vs. asset mode)
    \item Quadratic mean for conservative estimates
    \item Dynamic risk matrix application
\end{itemize}

\item \textbf{Data Management}:
\begin{itemize}
    \item CSV-based threat/asset databases
    \item Timestamped versioning of assessments
    \item Export to multiple formats (CSV, DOCX)
\end{itemize}

\item \textbf{User Interface Enhancements}:
\begin{itemize}
    \item Color-coded criteria for visual distinction
    \item Comprehensive mouse wheel handling
    \item Responsive scrollable layouts
\end{itemize}
\end{itemize}

\subsection{Implementation Details}

The tool's core functionality is implemented through several key components:

\begin{lstlisting}[language=Python, caption=Core Risk Calculation Methods]
def calculate_threat_likelihood(self, key):
"""Calculates Threat Likelihood combining threat-specific likelihood with asset likelihood"""
# Calculate from first 5 threat criteria (columns 0-4)
threat_values = []
for col_idx in [0, 1, 2, 3, 4]:
value_str = self.combo_vars[key][col_idx].get().strip()
if value_str:
threat_values.append(float(value_str))

# Quadratic mean calculation
threat_quadratic_mean = math.sqrt(sum(x**2 for x in threat_values) / len(threat_values))
threat_likelihood = (threat_quadratic_mean - 1) / 4  # Normalize [1,5]->[0,1]

# Combine with asset likelihood using risk matrix
asset_likelihood = self.get_asset_likelihood_for_key(key)
if asset_likelihood >= 0:
    threat_cat = self.value_to_category(threat_likelihood)
    asset_cat = self.value_to_category(asset_likelihood)
    return self.RISK_MATRIX.get((threat_cat, asset_cat), threat_cat)
return threat_likelihood
def calculate_asset_impact(self, key):
"""Calculates Asset Impact using quadratic mean of last 5 criteria"""
values = []
for col_idx in [4, 5, 6, 7, 8]:
value_str = self.combo_vars[key][col_idx].get().strip()
if value_str:
values.append(float(value_str))

quadratic_mean = math.sqrt(sum(x**2 for x in values) / len(values))
impact = (quadratic_mean - 1) / 4  # Normalize [1,5]->[0,1]
return self.value_to_category(impact)
\end{lstlisting}

\subsection{User Interface}

The interface implements several professional enhancements:

\begin{itemize}
\item \textbf{Color-Coded Criteria}:
\begin{itemize}
\item Distinct colors for each assessment criterion
\item Visual legends for quick reference
\item Consistent styling across both threat and asset views
\end{itemize}

\item \textbf{Interactive Elements}:
\begin{itemize}
    \item Protected comboboxes with mouse wheel handling
    \item Dynamic risk updates on value changes
    \item Context-aware help systems
\end{itemize}

\item \textbf{Data Visualization}:
\begin{itemize}
    \item Color-coded risk levels in main table
    \item Clear distinction between calculated and input fields
    \item Responsive table layouts for large datasets
\end{itemize}
\end{itemize}

\begin{lstlisting}[language=Python, caption=Interface Configuration]
def setup_combobox_styles(self):
"""Configure custom styles for Comboboxes with criteria colors"""
style = ttk.Style()
for i, color in enumerate(self.CRITERIA_COLORS):
style_name = f"Criteria{i}.TCombobox"
style.configure(style_name,
fieldbackground=color,
foreground='black',
selectbackground=color)
\end{lstlisting}

\subsection{Outputs and Reporting}

The tool generates comprehensive outputs:

\begin{enumerate}
\item \textbf{Risk Matrices}:
\begin{itemize}
\item Consolidated threat/asset risk overviews
\item Maximum risk calculations per threat
\item Cross-referenced likelihood-impact pairs
\end{itemize}

\item \textbf{Detailed Reports}:
\begin{itemize}
    \item Threat assessment documentation
    \item Asset evaluation summaries
    \item Control recommendation mapping
\end{itemize}

\item \textbf{Data Exports}:
\begin{itemize}
    \item CSV format for further analysis
    \item Word reports for documentation
    \item Legacy format support
\end{itemize}
\end{enumerate}

\subsection{Technical Innovations}

The tool introduces several technical advancements:

\begin{itemize}
\item \textbf{Context-Aware Calculation}:
\begin{itemize}
\item Automatic detection of assessment mode (threat/asset)
\item Appropriate method selection based on context
\item Unified risk presentation
\end{itemize}

\item \textbf{Data Versioning}:
\begin{itemize}
    \item Timestamped assessment storage
    \item Automatic loading of latest assessments
    \item Clear distinction between imported and current data
\end{itemize}

\item \textbf{Error Handling}:
\begin{itemize}
    \item Comprehensive input validation
    \item Graceful degradation for missing data
    \item Detailed logging
\end{itemize}
\end{itemize}

\subsection{Dynamic Cybersecurity Controls Management}

The Comprehensive Risk Assessment Tool incorporates advanced cybersecurity controls management capabilities that enable dynamic integration of security measures with existing risk assessments. This feature addresses the critical need for evaluating the effectiveness of implemented controls while maintaining assessment integrity.

\subsubsection{Control Integration Architecture}

The controls management system employs a sophisticated framework based on 125 cybersecurity controls derived from 13 international standards:

\begin{itemize}
\item \textbf{Multi-Framework Control Database}: Integration of controls from ISO 27001, NIST Cybersecurity Framework 2.0, NIST IR 8270, SPARTA, BSI Profile for Space, and other space-specific guidelines
\item \textbf{Intelligent Compatibility Mapping}: Automatic assessment of control applicability based on asset segments (Ground, Space, Link, User, Human Resources) and lifecycle phases
\item \textbf{Threat-Control Correlation}: Systematic mapping of controls to specific threat categories and risk assessment criteria
\end{itemize}

\subsubsection{Real-Time Impact Analysis}

The system provides immediate feedback on control selection effectiveness through several analytical components:

\begin{lstlisting}[language=Python, caption=Control Impact Analysis Implementation]
def analyze_control_effectiveness(self, selected_controls):
    """Analyze real-time impact of selected cybersecurity controls"""
    
    criteria_improvements = {}
    threat_coverage = {}
    segment_coverage = {}
    
    for control_id in selected_controls:
        control = self.control_database[control_id]
        
        # Analyze criteria impact
        for criterion in control['affected_criteria']:
            if criterion not in criteria_improvements:
                criteria_improvements[criterion] = 0
            criteria_improvements[criterion] += control['effectiveness_score']
        
        # Analyze threat coverage
        for threat in control['addressed_threats']:
            if threat not in threat_coverage:
                threat_coverage[threat] = []
            threat_coverage[threat].append(control_id)
        
        # Analyze segment coverage
        for segment in control['applicable_segments']:
            if segment not in segment_coverage:
                segment_coverage[segment] = 0
            segment_coverage[segment] += 1
    
    return {
        'criteria_impact': criteria_improvements,
        'threat_coverage': threat_coverage,
        'segment_coverage': segment_coverage,
        'coverage_quality': self.assess_coverage_quality(threat_coverage)
    }
\end{lstlisting}

\subsubsection{Assessment Synchronization and Integrity Protection}

The framework implements sophisticated mechanisms to maintain assessment consistency when controls are applied:

\begin{itemize}
\item \textbf{Read-Only Mode Activation}: Once controls are applied, the threat assessment interface automatically transitions to read-only mode, preventing manual modifications that could conflict with control-based risk adjustments
\item \textbf{Baseline Preservation}: Original assessment values are preserved alongside control-modified values, enabling comparative analysis and validation of control effectiveness
\item \textbf{Reversible Integration}: Controls can be cleared to restore full editing capabilities, allowing iterative refinement of both assessments and control selections
\end{itemize}

\subsubsection{Control Application Methodology}

The control application process follows a systematic approach:

\begin{lstlisting}[language=Python, caption=Control Application Workflow]
def apply_controls_to_assessment(self, selected_controls):
    """Apply cybersecurity controls and recalculate risk assessments"""
    
    # Phase 1: Calculate control effectiveness per criteria
    control_adjustments = self.calculate_criteria_adjustments(selected_controls)
    
    # Phase 2: Apply adjustments to threat assessment data
    for threat_id, asset_adjustments in control_adjustments.items():
        for asset_id, criteria_improvements in asset_adjustments.items():
            self.apply_criteria_improvements(threat_id, asset_id, criteria_improvements)
    
    # Phase 3: Recalculate risk levels with control impacts
    self.recalculate_all_risk_levels()
    
    # Phase 4: Enable assessment protection mode
    self.enable_read_only_mode()
    
    # Phase 5: Generate control application report
    return self.generate_control_impact_report(selected_controls, control_adjustments)

def calculate_criteria_adjustments(self, controls):
    """Calculate how controls improve specific risk assessment criteria"""
    adjustments = {}
    
    for control in controls:
        for threat in control['addressed_threats']:
            for asset in control['applicable_assets']:
                for criterion, improvement in control['criteria_improvements'].items():
                    # Apply segment-specific filtering
                    if self.is_control_applicable(control, asset):
                        self.add_adjustment(adjustments, threat, asset, criterion, improvement)
    
    return adjustments
\end{lstlisting}

\subsubsection{User Interface Enhancements}

The controls management interface incorporates several advanced features:

\begin{itemize}
\item \textbf{Intelligent Search and Filtering}: Real-time filtering by threat names, control descriptions, effectiveness criteria, and asset segments
\item \textbf{Dynamic Impact Visualization}: Live display of control selection effects on threat coverage, criteria improvements, and risk level changes
\item \textbf{Coverage Analysis Dashboard}: Real-time assessment of threat coverage quality with color-coded indicators for Excellent (4+ controls), Good (2-3 controls), and Basic (1 control) coverage levels
\item \textbf{Mouse Wheel Protection}: Enhanced scrolling behavior that prevents accidental modification of control selections during navigation
\end{itemize}

\subsection{Integration Capabilities}

The tool supports several integration features:

\begin{itemize}
\item \textbf{Data Import/Export}:
\begin{itemize}
\item CSV-based threat/asset definitions
\item Word report generation
\item Legacy data import
\end{itemize}

\item \textbf{Cross-Tool Compatibility}:
\begin{itemize}
    \item Consistent data formats with Phase 0/A tool
    \item Shared risk calculation methodologies
    \item Compatible reporting structures
\end{itemize}

\item \textbf{Extensible Architecture}:
\begin{itemize}
    \item Modular design for future enhancements
    \item Clear separation of calculation and UI layers
    \item Well-defined API for integration
\end{itemize}
\end{itemize}




\section{Attack Graph Analysis Tool}
\subsection{Purpose and Scope}

The Attack Graph Analysis Tool represents a significant advancement in space systems cybersecurity risk assessment. Designed to model complex threat relationships, this tool enables analysts to:

\begin{itemize}
\item Visualize cascading cyber threats across interconnected space system components
\item Quantify attack path criticality through multi-dimensional scoring metrics
\item Identify systemic vulnerabilities that traditional risk assessment methods might overlook
\item Support risk-based decision making for space mission architects and security engineers
\end{itemize}

The tool's analytical approach is particularly valuable for:

\begin{itemize}
\item Mission-critical systems where single points of failure could have catastrophic consequences
\item Complex system-of-systems architectures common in modern space infrastructure
\item Supply chain risk assessment across multi-vendor ecosystems
\end{itemize}

\subsection{Methodological Framework}

The tool implements a rigorous six-phase methodology:

\begin{enumerate}
\item \textbf{Threat Data Ingestion}:
\begin{itemize}
\item Parses CSV inputs with threat relationships (source threat, target threat, relation type)
\item Supports optional threat subset filtering through dedicated configuration files
\item Validates data integrity through schema verification
\end{itemize}

\item \textbf{Graph Construction}:
\begin{itemize}
    \item Builds directed graph using NetworkX library
    \item Nodes represent threats with category attributes
    \item Edges encode relationship types (enables, causes, leads-to, etc.)
\end{itemize}

\item \textbf{Dynamic Configuration}:
\begin{itemize}
    \item Automatically adjusts analysis parameters based on graph characteristics
    \item Identifies critical nodes through centrality metrics
    \item Optimizes path analysis depth based on graph density
\end{itemize}

\item \textbf{Multi-Layer Analysis}:
\begin{itemize}
    \item Performs topological analysis of attack surfaces
    \item Computes centrality measures (degree, betweenness, PageRank)
    \item Identifies critical paths using hybrid scoring (likelihood, impact, path complexity)
\end{itemize}

\item \textbf{Visualization}:
\begin{itemize}
    \item Generates hierarchical layouts for attack paths
    \item Implements color-coded node categorization
    \item Produces publication-quality outputs in multiple formats
\end{itemize}

\item \textbf{Reporting}:
\begin{itemize}
    \item Generates comprehensive textual analysis
    \item Outputs ranked threat lists with justification
    \item Documents critical paths with mitigation recommendations
\end{itemize}
\end{enumerate}

\subsection{Core Analytical Capabilities}

The tool implements several innovative analysis techniques:

\subsubsection{Adaptive Path Scoring}

Each attack path receives a composite criticality score combining:

\begin{equation}
S_p = \alpha L + \beta I + \gamma C + \delta D
\end{equation}

Where:
\begin{itemize}
\item $L$: Path likelihood (based on edge relationship types)
\item $I$: Cumulative impact (from threat attributes)
\item $C$: Category diversity (penalizes mono-category paths)
\item $D$: Depth factor (weights path length appropriately)
\end{itemize}

\subsubsection{Dynamic Node Prioritization}

The tool automatically identifies:

\begin{itemize}
\item \textbf{High-Value Targets}: Nodes with high in-degree and impact scores
\item \textbf{Attack Multipliers}: Nodes with high out-degree and likelihood scores
\item \textbf{Systemic Chokepoints}: Nodes with high betweenness centrality
\end{itemize}

\subsubsection{Threat Propagation Analysis}

The tool models how threats propagate through the system by:

\begin{itemize}
\item Calculating predecessor/successor networks for critical threats
\item Identifying second-order effects (neighbors of neighbors)
\item Evaluating threat clustering within system categories
\end{itemize}

\subsection{Implementation Highlights}

The Python implementation demonstrates several sophisticated features:

\begin{lstlisting}[language=Python, caption=Core Analysis Logic]
def _calculate_path_criticality(self, path, high_risk_threats=None):
"""Calculate composite criticality score for attack path"""
score = 0

# 1. Path length factor (longer paths are more complex)
length_factor = len(path) * 0.5

# 2. Relation type weights
relation_weights = {'Enables':3, 'Causes':4, 'Leads-to':2, 
                   'Triggers':3, 'Amplifies':2}
relation_score = sum(
    relation_weights.get(self.graph[path[i]][path[i+1]].get(
        'relation_type','Unknown'),1) 
    for i in range(len(path)-1)
)

# 3. Node criticality (matches against high-risk threats)
node_criticality = sum(
    1 for node in path 
    if any(threat.lower() in node.lower() 
          for threat in (high_risk_threats or self._get_top_risk_threats()))
)

# 4. Category diversity
categories = {self.graph.nodes[node].get('category','Unknown') 
             for node in path}
category_diversity = len(categories) * 0.5

return length_factor + relation_score + node_criticality + category_diversity
\end{lstlisting}

\subsection{Visualization Approach}

The tool generates three primary visualization types:

\begin{enumerate}
\item \textbf{System-Wide Attack Graph}:
\begin{itemize}
\item Spring or hierarchical layout
\item Color-coded by threat category
\item Edge styling by relation type
\end{itemize}

\item \textbf{Threat-Specific Star Graphs}:
\begin{itemize}
    \item Radial layout centered on critical threats
    \item Distinguishes predecessors/successors
    \item Shows relationship types
\end{itemize}

\item \textbf{Path-Specific Diagrams}:
\begin{itemize}
    \item Linear flow from source to target
    \item Color gradients indicate criticality
    \item Annotated with relation details
\end{itemize}
\end{enumerate}

\begin{lstlisting}[language=Python, caption=Visualization Configuration]
def _create_hierarchical_threat_connections_layout(self, graph, central_threat,
predecessors, successors):
"""Create optimized layout for threat connection visualization"""
pos = {}
center_x, center_y = 0, 0 # Central threat position

# Position predecessors to the left with distance-based spacing
left_levels = self._organize_nodes_by_distance(graph, central_threat, predecessors, reverse=True)
for level, nodes in left_levels.items():
    x_pos = -8 - (level * 2)  # Progressive left offset
    y_positions = np.linspace(-5, 5, len(nodes))  # Vertical distribution
    for node, y in zip(nodes, y_positions):
        pos[node] = (x_pos, y)

# Position successors to the right 
right_levels = self._organize_nodes_by_distance(graph, central_threat, successors, reverse=False)
for level, nodes in right_levels.items():
    x_pos = 8 + (level * 2)  # Progressive right offset
    y_positions = np.linspace(-5, 5, len(nodes))
    for node, y in zip(nodes, y_positions):
        pos[node] = (x_pos, y)

pos[central_threat] = (center_x, center_y)
return pos
\end{lstlisting}

\subsection{Outputs and Reporting}

The tool generates comprehensive outputs:

\begin{itemize}
\item \textbf{Technical Reports}:
\begin{itemize}
\item Graph statistics (nodes, edges, density)
\item Centrality rankings
\item Critical path analysis
\end{itemize}

\item \textbf{Visual Artifacts}:
\begin{itemize}
    \item Publication-quality diagrams
    \item Interactive GEXF exports for Gephi
    \item Annotated path visualizations
\end{itemize}

\item \textbf{Actionable Recommendations}:
\begin{itemize}
    \item Priority mitigation targets
    \item Relationship types needing hardening
    \item Monitoring points for attack detection
\end{itemize}
\end{itemize}

\subsection{Validation and Case Studies}

The tool has demonstrated effectiveness through:

\begin{itemize}
\item \textbf{Spacecraft Platform Analysis}:
\begin{itemize}
\item Identified unexpected command chain vulnerabilities
\item Revealed critical paths from ground systems to flight software
\end{itemize}

\item \textbf{Ground Segment Assessment}:
\begin{itemize}
    \item Highlighted systemic risks in multi-mission control systems
    \item Quantified threat propagation between segregated networks
\end{itemize}

\item \textbf{Supply Chain Evaluation}:
\begin{itemize}
    \item Mapped vulnerability propagation across vendor boundaries
    \item Identified critical dependencies on single-source components
\end{itemize}
\end{itemize}

\subsection{Limitations and Future Work}

Current limitations being addressed:

\begin{itemize}
\item \textbf{Computational Complexity}:
\begin{itemize}
\item Path finding becomes expensive for dense graphs (>500 nodes)
\item Implementing sampling techniques for large graphs
\end{itemize}

\item \textbf{Threat Intelligence Integration}:
\begin{itemize}
    \item Developing connectors to STIX/TAXII feeds
    \item Automating CVSS score incorporation
\end{itemize}

\item \textbf{Temporal Analysis}:
\begin{itemize}
    \item Adding mission phase awareness
    \item Modeling time-dependent vulnerabilities
\end{itemize}
\end{itemize}

% ============================================================================
% CHAPTER 6: VALIDATION OF FRAMEWORK AND TOOLS
% ============================================================================
\chapter{Validation of Framework and Tools}
\label{ch:validation}

This chapter presents the methodology and results of the validation process for the standardized risk assessment framework and the associated tool suite. The validation process demonstrates the technical robustness, operational effectiveness, and practical applicability of the developed framework and tools through comprehensive testing and iterative improvement.

\section{Technical Validation Methodology}

The validation methodology focused on technical robustness and operational effectiveness through systematic testing:

\begin{enumerate}
    \item \textbf{Functional Testing}: Comprehensive testing of all tool functions across different operational scenarios
    \item \textbf{Integration Testing}: Validation of data flow and compatibility between tool components
    \item \textbf{Environment Testing}: Testing across development and compiled executable environments
    \item \textbf{Error Handling Validation}: Systematic testing of error conditions and recovery mechanisms
    \item \textbf{Performance Evaluation}: Assessment of tool performance under various load conditions
    \item \textbf{User Interface Testing}: Evaluation of interface usability and user experience
\end{enumerate}

\section{Development and Testing Process}

The validation process involved iterative development with continuous testing and improvement:

\subsection{Asset Data Management Validation}

The transition from static to dynamic asset loading was validated through comprehensive testing:

\begin{itemize}
    \item \textbf{Data Loading Verification}: Successful loading of 34 assets across 9 category combinations from Asset.csv
    \item \textbf{Error Handling Testing}: Validation of graceful fallback mechanisms when CSV files are unavailable
    \item \textbf{Data Integrity Checks}: Verification of asset data consistency across all tool components
    \item \textbf{Performance Impact Assessment}: Minimal performance impact from dynamic loading implementation
\end{itemize}

\subsection{Executable Environment Validation}

Extensive testing was conducted to ensure proper operation in compiled executable environments:

\begin{itemize}
    \item \textbf{Path Resolution Testing}: Validation of file path handling in both development and executable contexts
    \item \textbf{Data File Inclusion}: Verification of proper packaging and access to required data files
    \item \textbf{Inter-tool Communication}: Testing of tool launching and coordination through main interface
    \item \textbf{Resource Management}: Validation of proper resource allocation and cleanup in executable format
\end{itemize}

\section{Tool Suite Component Validation}

The validation process systematically tested each component of the tool suite:

\subsection{BID Phase Assessment Tool Validation}

\begin{itemize}
    \item \textbf{Interface Functionality}: All user interface elements tested for proper operation and responsiveness
    \item \textbf{Risk Calculation Logic}: Mathematical models validated for accuracy and consistency
    \item \textbf{Report Generation}: Export functionality tested across multiple formats and file sizes
    \item \textbf{Error Handling}: Robust error handling validated under various failure conditions
\end{itemize}

\subsection{Preliminary Risk Assessment Tool Validation}

\begin{itemize}
    \item \textbf{Asset Loading Integration}: Dynamic CSV asset loading validated with comprehensive error handling
    \item \textbf{Threat Analysis Workflow}: Complete threat assessment process tested for logical consistency
    \item \textbf{Data Import/Export}: Legacy data integration and modern export capabilities verified
    \item \textbf{User Documentation}: Comprehensive help system tested for completeness and accessibility
\end{itemize}

\subsection{Comprehensive Risk Assessment Tool Validation}

\begin{itemize}
    \item \textbf{Advanced Analytics}: Complex risk calculation algorithms validated for mathematical accuracy
    \item \textbf{Multi-Modal Assessment}: Both threat and asset assessment workflows thoroughly tested
    \item \textbf{Professional Interface}: Enhanced user interface components validated for usability
    \item \textbf{Integration Capabilities}: Data flow between assessment phases verified for consistency
\end{itemize}

\subsection{Attack Graph Analysis Tool Validation}

\begin{itemize}
    \item \textbf{Graph Construction}: Automated attack graph generation tested with various threat scenarios
    \item \textbf{Relationship Analysis}: Threat-asset relationship mapping validated for logical consistency
    \item \textbf{Visualization Quality}: Graph visualization components tested for clarity and informativeness
    \item \textbf{Export Functionality}: Multiple export formats validated for compatibility and quality
\end{itemize}

\section{Technical Validation Results}

The comprehensive technical validation process yielded the following results:

\subsection{Functional Reliability}

\begin{itemize}
    \item \textbf{Core Functionality}: All primary assessment functions operate reliably across different operational scenarios
    \item \textbf{Data Processing}: Asset loading, risk calculation, and report generation functions perform consistently
    \item \textbf{Error Recovery}: Robust error handling ensures continued operation even when individual components encounter issues
    \item \textbf{Cross-Tool Integration}: Seamless data flow and compatibility maintained between all tool components
\end{itemize}

\subsection{Performance Characteristics}

\begin{itemize}
    \item \textbf{Response Time}: User interface remains responsive even during complex calculations and large data operations
    \item \textbf{Memory Usage}: Efficient memory management prevents resource exhaustion during extended use
    \item \textbf{File Operations}: Asset loading and export operations complete within acceptable time frames
    \item \textbf{Scalability}: Tools handle varying complexity levels without significant performance degradation
\end{itemize}

\subsection{User Experience Validation}

\begin{itemize}
    \item \textbf{Interface Consistency}: Uniform design patterns and interaction models across all tools enhance usability
    \item \textbf{Documentation Effectiveness}: Comprehensive help systems provide adequate guidance for effective tool utilization
    \item \textbf{Workflow Logic}: Assessment processes follow logical sequences that support efficient completion
    \item \textbf{Error Communication}: Clear, actionable error messages help users understand and resolve issues
\end{itemize}

\subsection{Technical Robustness}

\begin{itemize}
    \item \textbf{Environment Compatibility}: Tools operate correctly in both development and compiled executable environments
    \item \textbf{Character Encoding}: ASCII-compatible implementation ensures reliable operation across different system configurations
    \item \textbf{File Path Resolution}: Robust path handling supports operation from various installation locations
    \item \textbf{Dependency Management}: Proper packaging ensures all required components are available in deployed executables
\end{itemize}

\section{Development Lessons Learned}

The validation and iterative development process provided valuable insights:

\subsection{Technical Implementation Insights}

\begin{itemize}
    \item \textbf{Character Encoding Criticality}: Unicode character compatibility represents a critical consideration for cross-platform executable deployment
    \item \textbf{Path Resolution Complexity}: Proper file path handling requires careful consideration of different execution environments
    \item \textbf{Data Centralization Benefits}: Centralized asset management significantly improves maintainability and consistency
    \item \textbf{Error Handling Importance}: Comprehensive error handling is essential for user confidence and operational reliability
\end{itemize}

\subsection{User Experience Considerations}

\begin{itemize}
    \item \textbf{Interface Consistency Value}: Uniform design patterns across tools significantly enhance user adoption and efficiency
    \item \textbf{Documentation Necessity}: Comprehensive help systems are essential for effective tool utilization in professional environments
    \item \textbf{Workflow Optimization}: Logical grouping of functions and intuitive navigation improve assessment completion rates
    \item \textbf{Visual Design Impact}: Professional appearance and branding enhance credibility in organizational settings
\end{itemize}

\subsection{Development Process Recommendations}

\begin{itemize}
    \item \textbf{Early Testing Importance}: Testing in target deployment environments early in development prevents late-stage critical issues
    \item \textbf{Systematic Validation}: Comprehensive testing protocols identify issues that might not be apparent during development
    \item \textbf{Iterative Improvement}: Continuous refinement based on testing results significantly improves final product quality
    \item \textbf{Robust Architecture}: Modular design patterns facilitate easier maintenance and future enhancements
\end{itemize}

\section{Framework Applicability Assessment}

\subsection{Scope and Limitations}

The validation process confirmed the framework's applicability within defined scope:

\begin{itemize}
    \item \textbf{Space System Focus}: The framework is specifically optimized for space system cybersecurity assessment
    \item \textbf{Lifecycle Coverage}: All major project phases from BID through operational phases are addressed
    \item \textbf{Scalability Range}: The framework accommodates projects of varying size and complexity levels
    \item \textbf{Standardization Achievement}: Consistent assessment approaches enable cross-project comparison and organizational learning
\end{itemize}

\subsection{Future Enhancement Opportunities}

The validation process identified opportunities for future development:

\begin{itemize}
    \item \textbf{Advanced Analytics}: Integration of machine learning techniques for enhanced threat prediction
    \item \textbf{Real-time Integration}: Development of interfaces for operational monitoring and incident response
    \item \textbf{Regulatory Alignment}: Enhanced mapping to evolving cybersecurity standards and regulations
    \item \textbf{Collaborative Features}: Multi-user assessment capabilities for distributed teams
\end{itemize}

% ============================================================================
% CHAPTER 7: USER MANUAL AND OPERATIONAL GUIDANCE
% ============================================================================
\chapter{User Manual and Operational Guidance}
\label{ch:user_manual}

This chapter provides comprehensive operational guidance for utilizing the Risk Assessment Tool Suite developed in this research. The manual covers all four integrated tools, detailing their actual functions, user interfaces, and practical application scenarios based on the implemented functionality. Each tool is presented with step-by-step instructions that reflect the real implementation and capabilities.

The Risk Assessment Tool Suite consists of four specialized tools designed to support cybersecurity professionals throughout the space project lifecycle: BID Phase Assessment, Preliminary Risk Assessment (Phase 0/A), Comprehensive Risk Assessment (Phases B-C-D), and Attack Graph Analysis.

\section{System Requirements and Installation}

\subsection{Technical Requirements}

The Risk Assessment Tool Suite requires:

\begin{itemize}
    \item \textbf{Operating System}: Windows 7 or later
    \item \textbf{Memory}: Minimum 4 GB RAM
    \item \textbf{Storage}: 200 MB available disk space
    \item \textbf{Dependencies}: All tools are standalone executables requiring no additional software
\end{itemize}

\subsection{Installation and Setup}

The tool suite is distributed as standalone executable files:

\begin{enumerate}
    \item Extract all files to a dedicated directory
    \item Ensure data files (Asset.csv, Threat.csv, Control.csv, attack\_graph\_threat\_relations.csv) are present
    \item The Output directory will be created automatically for report generation
    \item Launch tools by executing the corresponding .exe files or run \_Main.exe for the integrated launcher
\end{enumerate}

\section{Main Launcher Interface}

The Risk Assessment Tool Suite includes a central launcher (\_Main.exe) that provides unified access to all four tools:

\begin{figure}[H]
    \centering
    \includegraphics[width=\textwidth]{placeholder_main_launcher.png}
    \caption{Main Launcher Interface for Risk Assessment Tool Suite}
    \label{fig:main_launcher}
\end{figure}

The launcher displays:
\begin{itemize}
    \item \textbf{Tool Cards}: Visual representation of each tool with descriptions
    \item \textbf{Process Management}: Track running tools and their status
    \item \textbf{Quick Access}: Direct launch buttons for each assessment tool
    \item \textbf{Documentation Links}: Access to help and guidance resources
\end{itemize}

\section{BID Phase Assessment Tool (0-BID.exe)}

\subsection{Purpose and Functionality}

The BID Phase Assessment Tool calculates initial cybersecurity risk values for space projects during the proposal phase. It evaluates eleven key cybersecurity criteria using a weighted scoring methodology.

\subsection{Main Interface}

\begin{figure}[H]
    \centering
    \includegraphics[width=\textwidth]{placeholder_bid_interface.png}
    \caption{BID Phase Assessment Tool Interface}
    \label{fig:bid_interface}
\end{figure}

The interface consists of:
\begin{itemize}
    \item \textbf{Criteria Table}: Displays the eleven cybersecurity criteria with scoring guidelines
    \item \textbf{Assessment Grid}: Input fields for scoring each criterion (1-4 scale)
    \item \textbf{Weight Configuration}: Shows predefined weights for each criterion
    \item \textbf{Results Panel}: Real-time calculation of total risk score and risk level
    \item \textbf{Action Buttons}: Save, Export, Clear, and Help functionality
\end{itemize}

\subsection{Assessment Process}

\subsubsection{Step 1: Review Criteria}
The tool presents eleven cybersecurity criteria:
\begin{enumerate}
    \item Cybersecurity Requirements (Weight: 0.15)
    \item Security Architecture Constraints (Weight: 0.12)
    \item Cryptographic Requirements (Weight: 0.10)
    \item Authentication \& Access Control (Weight: 0.08)
    \item Supply Chain Security (Weight: 0.12)
    \item Threat Modeling Guidelines (Weight: 0.08)
    \item Security Compliance References (Weight: 0.07)
    \item Security Validation Requirements (Weight: 0.10)
    \item Incident Response Expectations (Weight: 0.05)
    \item Data Protection and Privacy (Weight: 0.07)
    \item Cybersecurity Historical Data (Weight: 0.06)
\end{enumerate}

\subsubsection{Step 2: Score Assessment}
For each criterion, assign a score from 1-4 based on the detailed scoring guidelines:
\begin{itemize}
    \item \textbf{Score 1 (Low)}: Comprehensive requirements with clear specifications
    \item \textbf{Score 2 (Significant)}: Partial requirements with some details
    \item \textbf{Score 3 (Moderate)}: Basic requirements with limited detail
    \item \textbf{Score 4 (High)}: Missing or inadequate requirements
\end{itemize}

\subsubsection{Step 3: Calculate Results}
The tool automatically calculates:
\begin{itemize}
    \item Weighted total score (0.0-1.0 range)
    \item Risk level classification:
    \begin{itemize}
        \item 0.0-0.1: Very Low
        \item 0.1-0.4: Low
        \item 0.4-0.7: Medium
        \item 0.7-0.9: High
        \item 0.9-1.0: Very High
    \end{itemize}
\end{itemize}

\subsection{Export Functionality}

The BID tool provides an export option:
\begin{itemize}
    \item \textbf{Word Report}: Professional formatted assessment report
\end{itemize}

\section{Preliminary Risk Assessment Tool (1-Risk\_Assessment\_0-A.exe)}

\subsection{Purpose and Functionality}

The Preliminary Risk Assessment Tool performs threat-centric analysis for Phase 0/A space projects using a five-criteria assessment framework. It evaluates threats against asset categories to calculate likelihood, impact, and risk levels.

\subsection{Main Interface}

\begin{figure}[H]
    \centering
    \includegraphics[width=\textwidth]{placeholder_prelim_interface.png}
    \caption{Preliminary Risk Assessment Tool Main Interface}
    \label{fig:prelim_interface}
\end{figure}

The main interface displays:
\begin{itemize}
    \item \textbf{Threat Summary Table}: Overview of all threats with calculated risk levels
    \item \textbf{Asset Categories}: Loaded from Asset.csv (categories and subcategories only)
    \item \textbf{Assessment Buttons}: Access to detailed threat analysis
    \item \textbf{Export/Import Controls}: Data management and reporting functions
\end{itemize}

\subsection{Assessment Workflow}

\subsubsection{Step 1: Threat Selection}
Click "THREAT ANALYSIS" to access the detailed assessment interface:

\begin{figure}[H]
    \centering
    \includegraphics[width=\textwidth]{placeholder_prelim_threat_assessment.png}
    \caption{Threat Assessment Interface with Five Criteria}
    \label{fig:prelim_threat_assessment}
\end{figure}

\subsubsection{Step 2: Criteria Evaluation}
For each threat-asset combination, evaluate using five criteria (1-5 scale):
\begin{enumerate}
    \item \textbf{Vulnerability Level}: Assessment of known vulnerabilities
    \item \textbf{Detection Probability}: Likelihood that malicious activities will be detected
    \item \textbf{Defense Capability}: Comprehensive defense including mitigations, access controls, and privilege requirements
    \item \textbf{Operational Impact}: Assessment of mission impact
    \item \textbf{Recovery Time}: Evaluation of recovery capabilities
\end{enumerate}

\subsubsection{Step 3: Automatic Risk Calculation}
The tool automatically calculates:
\begin{itemize}
    \item \textbf{Likelihood}: Quadratic mean of first three criteria
    \item \textbf{Impact}: Quadratic mean of last two criteria
    \item \textbf{Risk Level}: Using ISO 27005 risk matrix
\end{itemize}

\subsection{Data Management Features}

\begin{itemize}
    \item \textbf{Save Assessment}: Store current work temporarily
    \item \textbf{Export Report}: Generate final Word documentation
    \item \textbf{Import Report}: Load previously saved assessments
    \item \textbf{Import Legacy}: Import data from previous Phase 0/A assessments
\end{itemize}

\section{Comprehensive Risk Assessment Tool (2-Risk\_Assessment.exe)}

\subsection{Purpose and Functionality}

The Comprehensive Risk Assessment Tool provides advanced dual-mode analysis for Phases B-C-D, supporting both threat-centric and asset-centric assessment approaches with detailed criteria frameworks.

\subsection{Main Interface}

\begin{figure}[H]
    \centering
    \includegraphics[width=\textwidth]{placeholder_comp_main.png}
    \caption{Comprehensive Risk Assessment Tool Main Interface}
    \label{fig:comp_main}
\end{figure}

The main interface shows:
\begin{itemize}
    \item \textbf{Threat Risk Assessment Table}: Summary of all threats with risk levels
    \item \textbf{Analysis Mode Buttons}: Access to Threat Analysis and Asset Analysis
    \item \textbf{Export Controls}: Comprehensive reporting and data management
\end{itemize}

\subsection{Threat Analysis Mode}

\subsubsection{Interface and Workflow}

\begin{figure}[H]
    \centering
    \includegraphics[width=\textwidth]{placeholder_comp_threat_analysis.png}
    \caption{Threat Analysis Mode Interface}
    \label{fig:comp_threat_analysis}
\end{figure}

The threat analysis mode includes:
\begin{enumerate}
    \item \textbf{Criteria Reference Table}: Seven-criteria framework with detailed descriptions
    \item \textbf{Threat Selection}: Dropdown menu to select specific threats
    \item \textbf{Asset Assessment Grid}: Evaluation of all assets for the selected threat
    \item \textbf{Real-time Calculations}: Automatic likelihood, impact, and risk computation
\end{enumerate}

\subsubsection{Seven-Criteria Framework}
For each threat-asset combination, evaluate:
\begin{enumerate}
    \item \textbf{Vulnerability Effectiveness}: Exploitability assessment
    \item \textbf{Mitigation Presence}: Countermeasure effectiveness
    \item \textbf{Detection Probability}: Detection capability assessment
    \item \textbf{Access Complexity}: Access difficulty evaluation
    \item \textbf{Privilege Requirement}: Required privilege level
    \item \textbf{Response Delay}: Response time capability
    \item \textbf{Resilience Impact}: Operational resilience assessment
\end{enumerate}

\subsection{Asset Analysis Mode}

\subsubsection{Interface and Functionality}

\begin{figure}[H]
    \centering
    \includegraphics[width=\textwidth]{placeholder_comp_asset_analysis.png}
    \caption{Asset Analysis Mode Interface}
    \label{fig:comp_asset_analysis}
\end{figure}

The asset analysis mode provides:
\begin{enumerate}
    \item \textbf{Nine-Criteria Framework}: Comprehensive asset evaluation
    \item \textbf{Independent Assessment}: Asset analysis separate from threat context
    \item \textbf{Dual Calculation}: Both likelihood and impact from asset perspective
\end{enumerate}

\subsubsection{Nine-Criteria Asset Framework}
Asset evaluation using:

\textbf{Likelihood Criteria (1-4):}
\begin{enumerate}
    \item \textbf{Dependency}: Asset criticality to mission operations
    \item \textbf{Penetration}: Potential access levels achievable
    \item \textbf{Cyber Maturity}: Organizational security maturity
    \item \textbf{Trust}: Stakeholder trustworthiness assessment
\end{enumerate}

\textbf{Impact Criteria (5-9):}
\begin{enumerate}
    \setcounter{enumi}{4}
    \item \textbf{Performance}: Operational performance impact
    \item \textbf{Schedule}: Project timeline implications
    \item \textbf{Costs}: Financial impact assessment
    \item \textbf{Reputation}: Reputational consequences
    \item \textbf{Recovery}: Recovery time and effort required
\end{enumerate}

\subsection{Advanced Features}

\subsubsection{Integrated Risk Calculation}
The tool combines threat and asset assessments by:
\begin{itemize}
    \item Using threat-specific criteria for threat analysis
    \item Incorporating asset assessment data when available
    \item Combining impact calculations using risk matrix operations
    \item Providing comprehensive risk visibility
\end{itemize}

\subsubsection{Data Management}
\begin{itemize}
    \item \textbf{Save Assessment}: Temporary storage of work in progress
    \item \textbf{Export Report}: Generate final Word documentation with comprehensive tables
    \item \textbf{Export CSV}: Detailed data export for further analysis
    \item \textbf{Import Report}: Load previously saved comprehensive assessments
    \item \textbf{Import Report 0-A}: Import data from Preliminary Risk Assessment
\end{itemize}

\subsection{Controls Management System}

The Comprehensive Risk Assessment Tool includes an advanced Controls Management system that enables dynamic integration of cybersecurity controls with risk assessments. This feature is particularly valuable for evaluating the effectiveness of implemented security measures.

\subsubsection{Accessing Controls Management}

To access the Controls Management system:
\begin{enumerate}
    \item Complete your baseline threat or asset assessment
    \item Click the \textbf{"CONTROLS MANAGEMENT"} button on the main interface
    \item The Controls Management window opens with a dual-panel layout
\end{enumerate}

\begin{figure}[H]
    \centering
    \includegraphics[width=\textwidth]{placeholder_controls_management.png}
    \caption{Controls Management Interface - Dual Panel Layout}
    \label{fig:controls_management}
\end{figure}

\subsubsection{Interface Layout and Navigation}

The Controls Management interface features:

\begin{itemize}
    \item \textbf{Left Panel (65\% width)}: Control selection and browsing interface
    \item \textbf{Right Panel (35\% width)}: Real-time impact analysis dashboard
    \item \textbf{Bottom Button Bar}: Clear All, Help, and Save \& Apply controls
\end{itemize}

\paragraph{Control Selection Panel}
The left panel provides comprehensive control browsing capabilities:

\begin{itemize}
    \item \textbf{Search Bar}: Filter controls by threat names, descriptions, or effectiveness criteria
    \item \textbf{Control Clusters}: Organized categories including Policies \& Procedures, Risk Management, Security by Design, Network Security, and others
    \item \textbf{Control Details}: Each control shows ID, title, description, and compatibility information
    \item \textbf{Segment Compatibility}: Visual indicators for Ground, Space, Link, User, and Human Resources segments
\end{itemize}

\paragraph{Real-Time Impact Analysis Panel}
The right panel displays immediate feedback on control selections:

\begin{itemize}
    \item \textbf{Criteria Impact}: Shows which risk assessment criteria will be improved
    \item \textbf{Threat Coverage}: Identifies threats addressed by selected controls
    \item \textbf{Coverage Quality}: Color-coded indicators for Excellent (4+ controls), Good (2-3 controls), Basic (1 control)
    \item \textbf{Control Summary}: Statistics on selected controls by cluster
\end{itemize}

\subsubsection{Control Selection Workflow}

\paragraph{Step 1: Browse and Search Controls}
\begin{enumerate}
    \item Use the search bar to filter controls by keywords (e.g., "jamming", "encryption", "detection")
    \item Expand control clusters to view all controls in specific categories
    \item Review control descriptions and compatibility information
    \item Identify controls relevant to your specific threats and assets
\end{enumerate}

\paragraph{Step 2: Select Appropriate Controls}
\begin{enumerate}
    \item Check the boxes next to controls you want to apply
    \item Monitor the real-time impact analysis as you make selections
    \item Ensure adequate coverage for high-risk threats
    \item Consider asset segment compatibility when selecting controls
\end{enumerate}

\paragraph{Step 3: Review Impact Analysis}
\begin{enumerate}
    \item Examine the Criteria Impact section to see which risk factors will be improved
    \item Review Threat Coverage to identify any gaps in protection
    \item Check Coverage Quality indicators to ensure adequate control density
    \item Verify Control Summary provides balanced coverage across security domains
\end{enumerate}

\paragraph{Step 4: Apply Controls}
\begin{enumerate}
    \item Click \textbf{"Save \& Apply Controls"} to integrate controls with your assessment
    \item The system recalculates all risk levels based on control effectiveness
    \item Threat Assessment automatically enters \textbf{READ-ONLY MODE}
    \item A confirmation dialog shows the number of applied controls
\end{enumerate}

\subsubsection{Understanding Control Database}

The system includes 125 cybersecurity controls derived from 13 international frameworks:

\begin{itemize}
    \item \textbf{Source Frameworks}: ISO 27001, NIST CSF 2.0, NIST IR 8270, SPARTA, BSI Profile for Space, and others
    \item \textbf{Control Categories}: 12 strategic clusters covering all aspects of cybersecurity
    \item \textbf{Lifecycle Mapping}: Controls mapped to appropriate mission phases (All, Phase B/C, Phase D, Phase E, Phase F)
    \item \textbf{Segment Compatibility}: Intelligent filtering based on asset segment applicability
\end{itemize}

\subsubsection{Asset Segment Compatibility System}

Controls are automatically filtered based on technical and operational compatibility:

\begin{itemize}
    \item \textbf{Ground Segment}: Ground stations, mission control, data processing centers
    \item \textbf{Space Segment}: Satellites, spacecraft, orbital platforms and payloads
    \item \textbf{Link Segment}: Communication channels between space and ground systems
    \item \textbf{User Segment}: End-user terminals and service consumption points
    \item \textbf{Human Resources}: Personnel controls, training, and organizational measures
\end{itemize}

\subsubsection{Read-Only Mode and Assessment Protection}

Once controls are applied, the system implements several protection mechanisms:

\begin{itemize}
    \item \textbf{Assessment Lock}: Threat Analysis interface becomes read-only to prevent conflicts
    \item \textbf{Baseline Preservation}: Original assessment values are maintained alongside control-modified values
    \item \textbf{Visual Indicators}: Clear interface indicators show read-only status
    \item \textbf{Consistent Integration}: Control impacts are automatically integrated with risk calculations
\end{itemize}

\subsubsection{Advanced Features and Best Practices}

\paragraph{Mouse Wheel Protection}
The interface includes enhanced scrolling behavior that prevents accidental modification of control selections during navigation.

\paragraph{Search and Filtering Tips}
\begin{itemize}
    \item Search by threat names (e.g., "denial", "jamming") to find relevant controls
    \item Use criteria keywords (e.g., "detection", "mitigation") for targeted selection
    \item Filter by control titles or descriptions for specific implementations
    \item Clear search to return to full control catalog
\end{itemize}

\paragraph{Coverage Analysis Best Practices}
\begin{itemize}
    \item Aim for "Good" or "Excellent" coverage for high-risk threats
    \item Balance controls across different security domains
    \item Consider cumulative effects of multiple controls on the same threat
    \item Review asset segment coverage to ensure comprehensive protection
\end{itemize}

\paragraph{Clearing Controls and Returning to Edit Mode}
\begin{enumerate}
    \item Click \textbf{"Clear All Controls"} to remove all applied controls
    \item Confirm the action when prompted
    \item Original assessment values are restored
    \item Threat Analysis interface returns to editable mode
    \item You can then modify assessments or apply different controls
\end{enumerate}

\subsubsection{Integration with Risk Assessment Workflow}

The Controls Management system integrates seamlessly with the assessment workflow:

\begin{enumerate}
    \item \textbf{Baseline Assessment}: Complete initial threat and asset assessments without controls
    \item \textbf{Control Selection}: Use Controls Management to identify and select appropriate security measures
    \item \textbf{Impact Analysis}: Review real-time feedback on control effectiveness
    \item \textbf{Control Application}: Apply controls to see updated risk levels
    \item \textbf{Final Reporting}: Generate reports showing both baseline and control-enhanced risk profiles
\end{enumerate}

This comprehensive controls management capability enables organizations to make informed decisions about cybersecurity investments and evaluate the effectiveness of their security posture throughout the space mission lifecycle.

\section{Attack Graph Analysis Tool (3-attack\_graph\_analyzer.exe)}

\subsection{Purpose and Functionality}

The Attack Graph Analysis Tool analyzes threat relationships and generates attack graphs for space systems cybersecurity. It processes threat data to identify attack paths and critical relationships between threats.

\subsection{Interface and Operation}

\begin{figure}[H]
    \centering
    \includegraphics[width=\textwidth]{placeholder_attack_graph.png}
    \caption{Attack Graph Analysis Tool Interface}
    \label{fig:attack_graph}
\end{figure}

The tool operates through:
\begin{enumerate}
    \item \textbf{File Selection}: Choose CSV file containing threat data for analysis
    \item \textbf{Relationship Processing}: Load threat relationships from attack\_graph\_threat\_relations.csv
    \item \textbf{Graph Generation}: Create network graphs showing threat interconnections
    \item \textbf{Analysis Output}: Generate statistical reports and visualizations
\end{enumerate}

\subsection{Analysis Features}

\subsubsection{Graph Analysis Capabilities}
\begin{itemize}
    \item Network topology analysis
    \item Critical path identification
    \item Centrality calculations
    \item Threat clustering analysis
\end{itemize}

\subsubsection{Visualization Outputs}
\begin{itemize}
    \item \textbf{PNG Graphics}: High-quality network diagrams
    \item \textbf{Interactive Visualizations}: Dynamic graph exploration
    \item \textbf{Statistical Reports}: Comprehensive analysis summaries
    \item \textbf{GEXF Exports}: Compatibility with external graph analysis tools
\end{itemize}

\section{Data Management and Workflow Integration}

\subsection{Save vs Export Functionality}

All tools implement a consistent data management approach:

\begin{itemize}
    \item \textbf{Save Assessment}: Temporarily stores work in progress in application memory
    \item \textbf{Export Report}: Generates final documentation (Word/CSV) and permanently saves analysis results
    \item \textbf{Important}: Save Assessment data is temporary - always use Export Report for permanent documentation
\end{itemize}

\subsection{Import Capabilities}

\begin{itemize}
    \item \textbf{Import Report}: Load previously exported comprehensive assessments
    \item \textbf{Import Report 0-A}: Import data from Preliminary Risk Assessment (Phase 0/A)
    \item \textbf{Import Mission Analysis Report}: Import data from older assessment formats
\end{itemize}

\subsection{Output Directory Structure}

All tools save outputs to the Output directory:
\begin{itemize}
    \item \textbf{Word Reports}: Professional assessment documentation
    \item \textbf{CSV Exports}: Detailed data for further analysis
    \item \textbf{Graph Visualizations}: PNG and interactive graph files
    \item \textbf{Analysis Reports}: Statistical summaries and findings
\end{itemize}

\section{Best Practices and Recommendations}

\subsection{Assessment Workflow}

\begin{enumerate}
    \item \textbf{Start with BID Phase}: Establish initial risk baseline
    \item \textbf{Proceed to Preliminary Assessment}: Detailed threat analysis for Phase 0/A
    \item \textbf{Conduct Comprehensive Assessment}: Advanced analysis for Phases B-C-D
    \item \textbf{Perform Attack Graph Analysis}: Understand threat relationships
\end{enumerate}

\subsection{Data Management}

\begin{itemize}
    \item Always use Export Report for permanent documentation
    \item Regularly backup assessment data
    \item Maintain version control of CSV data files
    \item Document assessment assumptions and rationale
\end{itemize}

\subsection{Quality Assurance}

\begin{itemize}
    \item Validate asset and threat data accuracy
    \item Review calculation results for logical consistency
    \item Cross-reference findings between assessment modes
    \item Involve multiple assessors for critical evaluations
\end{itemize}

% ============================================================================
% CHAPTER 8: AI TRAINING AND DEVELOPMENT
% ============================================================================
\chapter{AI-Powered Risk Assessment Methodology}
\label{ch:ai_training}

This chapter presents the development and implementation of an AI-powered risk assessment system that extends the capabilities of the Risk Assessment Tool Suite through automated threat analysis and risk evaluation. The system leverages large language models to provide intelligent, context-aware cybersecurity assessments for space missions.

\section{AI Integration Rationale and Objectives}

The integration of artificial intelligence into cybersecurity risk assessment addresses several fundamental challenges in the space domain:

\begin{itemize}
    \item \textbf{Scalability}: Manual risk assessment becomes impractical for complex satellite constellations and multi-phase missions
    \item \textbf{Consistency}: Human analysts may apply different interpretations to similar scenarios, leading to assessment variability
    \item \textbf{Comprehensive Coverage}: AI systems can systematically evaluate all possible asset-threat combinations without fatigue or oversight
    \item \textbf{Speed}: Automated analysis enables rapid risk assessment iterations during mission planning phases
    \item \textbf{Knowledge Synthesis}: AI can integrate vast amounts of cybersecurity knowledge and apply it contextually to specific mission profiles
\end{itemize}

The AI-powered system was designed to complement, not replace, human expertise by providing a structured, comprehensive foundation for risk analysis that cybersecurity professionals can review, validate, and refine.

\section{Technical Architecture and Implementation}

\subsection{Local AI Infrastructure Selection}

The system employs Ollama as the local AI inference engine, coupled with the Mistral 7B language model. This architecture was selected based on several technical and operational considerations:

\begin{itemize}
    \item \textbf{Data Sovereignty}: Local processing ensures that sensitive mission data never leaves the organizational environment
    \item \textbf{Operational Independence}: No reliance on external API services or internet connectivity during risk assessments
    \item \textbf{Cost Efficiency}: Eliminates recurring costs associated with cloud-based AI services
    \item \textbf{Customization Potential}: Local deployment allows for model fine-tuning with domain-specific data
    \item \textbf{Performance Predictability}: Consistent response times independent of network conditions or service availability
\end{itemize}

\subsubsection{Ollama Configuration}

Ollama serves as the inference server, providing a RESTful API interface for model interactions. The system configuration includes:

\begin{lstlisting}[language=Python, caption=Ollama Integration Configuration]
class BatchAIRiskAssessment:
    def __init__(self):
        self.ollama_url = "http://localhost:11434"
        self.model = "mistral:7b"
        
    def query_ollama(self, prompt):
        response = requests.post(
            f"{self.ollama_url}/api/generate",
            json={
                "model": self.model,
                "prompt": prompt,
                "stream": False,
                "options": {
                    "temperature": 0.3,
                    "num_predict": 2000,
                    "num_ctx": 3072,
                    "top_k": 40,
                    "top_p": 0.9
                }
            },
            timeout=3000
        )
\end{lstlisting}

\subsubsection{Model Parameter Optimization}

The Mistral 7B model parameters were optimized for cybersecurity risk assessment tasks:

\begin{itemize}
    \item \textbf{Temperature (0.3)}: Low temperature ensures deterministic, factual responses while maintaining some creativity for threat scenario analysis
    \item \textbf{Context Window (3072 tokens)}: Sufficient context to process mission descriptions, asset lists, and threat catalogs simultaneously
    \item \textbf{Prediction Limit (2000 tokens)}: Balanced between comprehensive analysis and computational efficiency
    \item \textbf{Top-K (40) and Top-P (0.9)}: Conservative sampling parameters to maintain response quality and relevance
\end{itemize}

\subsection{Batch Processing Architecture}

To ensure comprehensive analysis without response truncation, the system employs a batch processing approach that divides the risk assessment into four sequential phases:

\begin{enumerate}
    \item \textbf{Context Analysis}: Mission characterization and asset relevance identification
    \item \textbf{Threat Mapping}: Systematic threat identification for each relevant asset
    \item \textbf{Risk Matrix Generation}: Quantitative risk assessment for all asset-threat pairs
    \item \textbf{Control Recommendation}: Security control selection and prioritization
\end{enumerate}

This architecture prevents the common issue of incomplete analyses that occur when large language models reach their output limits during complex reasoning tasks.

\section{Prompt Engineering and Domain Adaptation}

\subsection{Structured Prompt Design}

The system employs carefully engineered prompts that incorporate domain expertise and ensure consistent, comprehensive outputs. Each prompt follows a structured format:

\begin{lstlisting}[language=Python, caption=Example Context Analysis Prompt Structure]
prompt = f"""You are a cybersecurity analyst for satellite systems. 
Analyze this satellite program:

PROGRAM: {program_description}
AVAILABLE ASSETS: {', '.join(assets_sample)}

Provide:
1. PROGRAM CONTEXT ANALYSIS
   - Key mission characteristics
   - Primary operational environments
   - Mission criticality level

2. RELEVANT ASSETS IDENTIFICATION
   - List ALL assets relevant for this specific program
   - Explain why each asset is relevant
   - Do NOT skip any relevant assets

Be concise but complete. Format with clear sections."""
\end{lstlisting}

\subsection{Cybersecurity Domain Knowledge Integration}

The prompts integrate established cybersecurity frameworks and space-specific threat knowledge:

\begin{itemize}
    \item \textbf{Asset Taxonomy}: References to the complete asset catalog from the Risk Assessment Tool Suite
    \item \textbf{Threat Catalog}: Integration with the standardized threat taxonomy developed in earlier chapters
    \item \textbf{Risk Methodology}: Consistent application of likelihood and impact assessment criteria
    \item \textbf{Control Framework}: Alignment with established cybersecurity control catalogs
\end{itemize}

\subsection{Quality Assurance Mechanisms}

Several mechanisms ensure output quality and completeness:

\begin{itemize}
    \item \textbf{Explicit Completeness Instructions}: Prompts include specific directives against truncation or abbreviation
    \item \textbf{Structured Output Formats}: Required formatting ensures consistent parsing and integration
    \item \textbf{Cross-Reference Validation}: Each phase references outputs from previous phases for consistency
    \item \textbf{Error Handling}: Robust exception handling for network issues, model errors, and malformed responses
\end{itemize}

\section{Data Integration and Processing Pipeline}

\subsection{CSV Data Source Integration}

The AI system integrates seamlessly with the existing Risk Assessment Tool Suite data sources:

\begin{lstlisting}[language=Python, caption=Data Integration Implementation]
def __init__(self):
    try:
        self.threats_df = pd.read_csv("Threat.csv", sep=';')
        self.assets_df = pd.read_csv("Asset.csv", sep=';') 
        self.controls_df = pd.read_csv("Control.csv", sep=';')
        print(f"Loaded {len(self.threats_df)} threats, "
              f"{len(self.assets_df)} assets, "
              f"{len(self.controls_df)} controls")
    except Exception as e:
        print(f"CSV loading error: {e}")
        sys.exit(1)
\end{lstlisting}

\subsection{Automated Report Generation}

The system automatically generates comprehensive risk assessment reports in Markdown format, facilitating easy integration into documentation workflows:

\begin{itemize}
    \item \textbf{Structured Sections}: Each report follows a consistent organizational structure
    \item \textbf{Metadata Tracking}: Complete traceability of analysis parameters and data sources
    \item \textbf{Timestamp Documentation}: Precise tracking of assessment execution time
    \item \textbf{Version Control Integration}: Text-based format enables version control and collaborative review
\end{itemize}

\section{Validation and Performance Evaluation}

\subsection{Assessment Accuracy Validation}

The AI system was designed to provide structured, comprehensive risk assessments that can be validated and refined by cybersecurity professionals. Due to the prototype nature of this implementation and the absence of established ground truth datasets for space cybersecurity risk assessment, formal quantitative validation remains an area for future work.

Initial qualitative evaluation demonstrates several promising characteristics:

\begin{itemize}
    \item \textbf{Systematic Coverage}: The AI system consistently identifies and analyzes all relevant asset-threat combinations from the defined taxonomies, addressing a common limitation in manual assessments where analysts may inadvertently overlook certain combinations
    \item \textbf{Structured Reasoning}: The system provides explicit justifications for risk level assignments, enabling transparent review and validation by domain experts
    \item \textbf{Consistency}: Repeated analyses of identical mission scenarios produce consistent results, eliminating variability that can occur in manual assessments due to analyst fatigue or subjective interpretation
    \item \textbf{Comprehensive Documentation}: All assessment steps are automatically documented, providing complete traceability of the analysis process
\end{itemize}

\textbf{Validation Methodology Considerations:}

Establishing rigorous validation methodologies for AI-powered cybersecurity risk assessment presents several challenges unique to the space domain:

\begin{itemize}
    \item \textbf{Limited Ground Truth Data}: Unlike other domains where historical incident data can provide validation baselines, space cybersecurity incidents are rarely disclosed publicly, limiting the availability of reference datasets
    \item \textbf{Expert Variability}: Manual assessments by different cybersecurity experts may legitimately reach different conclusions based on varying experience, risk tolerance, and analytical approaches
    \item \textbf{Context Sensitivity}: Risk assessments are inherently dependent on mission-specific context, organizational risk appetite, and threat landscape evolution
    \item \textbf{Temporal Validity}: Cybersecurity threats and mitigations evolve rapidly, requiring continuous validation and model updates
\end{itemize}

Future validation efforts should incorporate structured expert review protocols, comparative analysis across multiple assessment methodologies, and longitudinal studies to evaluate prediction accuracy against actual security incidents when such data becomes available.

\subsection{Performance Characteristics}

System performance metrics demonstrate the practical viability of the AI-powered approach for operational deployment:

\begin{itemize}
    \item \textbf{Processing Time}: Complete risk assessments typically require 3-8 minutes per mission scenario on modern workstation hardware, representing a significant time advantage over manual assessment approaches which can require several hours to achieve comparable comprehensiveness
    \item \textbf{Resource Utilization}: The system operates efficiently on standard workstation configurations with moderate CPU and memory requirements. Local deployment eliminates network dependencies and ensures consistent performance
    \item \textbf{Scalability}: The batch processing architecture enables linear scaling for multiple mission scenarios, making it practical to analyze entire mission portfolios or conduct comparative risk studies
    \item \textbf{Reliability}: The modular design and comprehensive error handling ensure robust operation across diverse mission profiles and input scenarios
\end{itemize}

\section{Automated Batch Processing System}

\subsection{Multi-Mission Assessment Capability}

An automated batch processing system was developed to enable systematic analysis of multiple mission scenarios:

\begin{lstlisting}[language=Python, caption=Batch Processing Implementation]
def run_batch_assessment_for_all():
    results = []
    total_start_time = time.time()
    
    for i, (name, description) in enumerate(examples.items(), 1):
        print(f"[{i}/{len(examples)}] Processing: {name}")
        
        start_time = time.time()
        result = subprocess.run([
            sys.executable, 
            "AI_Risk_Assessment_Batch_Fixed.py",
            description.strip()
        ], capture_output=True, text=True, timeout=3600)
        
        elapsed_time = time.time() - start_time
        # Process results and generate summary
\end{lstlisting}

\subsection{Operational Integration Features}

The batch system includes several features designed for operational environments:

\begin{itemize}
    \item \textbf{Progress Monitoring}: Real-time status updates during multi-mission processing
    \item \textbf{Error Recovery}: Graceful handling of individual assessment failures without terminating batch operations
    \item \textbf{Resource Management}: Intelligent spacing between assessments to prevent system overload
    \item \textbf{Comprehensive Reporting}: Detailed summary statistics and success metrics for batch operations
\end{itemize}

\section{Integration with Risk Assessment Tool Suite}

\subsection{Architectural Compatibility}

The AI system was designed to complement the existing Risk Assessment Tool Suite while maintaining architectural independence:

\begin{itemize}
    \item \textbf{Data Format Consistency}: Uses identical CSV schemas for seamless data exchange
    \item \textbf{Methodology Alignment}: Applies the same risk assessment principles and criteria
    \item \textbf{Output Format Compatibility}: Generates reports compatible with existing documentation workflows
    \item \textbf{Modular Deployment}: Can be deployed independently or as part of the integrated suite
\end{itemize}

\subsection{Workflow Enhancement}

The AI system enhances existing workflows through several integration points:

\begin{itemize}
    \item \textbf{Preliminary Assessment}: Rapid initial risk analysis to inform manual assessment focus areas
    \item \textbf{Completeness Verification}: Systematic verification that manual assessments have not overlooked threats
    \item \textbf{Consistency Checking}: Cross-validation of manual risk level assignments
    \item \textbf{Documentation Support}: Automated generation of assessment documentation templates
\end{itemize}

\section{Limitations and Future Enhancements}

\subsection{Current Limitations}

Several limitations were identified during development and validation:

\begin{itemize}
    \item \textbf{Domain Knowledge Boundaries}: Model performance depends on training data representation of space cybersecurity concepts
    \item \textbf{Context Sensitivity}: Limited ability to incorporate highly specific mission constraints or novel threat scenarios
    \item \textbf{Quantitative Precision}: Risk level assignments may lack the precision achievable through detailed manual analysis
    \item \textbf{Model Dependency}: System effectiveness is constrained by the capabilities of the underlying language model
\end{itemize}

\subsection{Enhancement Opportunities}

Future development could address these limitations through several approaches:

\begin{itemize}
    \item \textbf{Domain-Specific Fine-Tuning}: Training on space cybersecurity-specific datasets to improve domain knowledge
    \item \textbf{Hybrid Analysis}: Integration with quantitative risk modeling tools for enhanced precision
    \item \textbf{Continuous Learning}: Implementation of feedback mechanisms to improve assessment quality over time
    \item \textbf{Multi-Model Ensemble}: Combination of multiple AI models to improve robustness and coverage
\end{itemize}

\section{Conclusions}

The AI-powered risk assessment system successfully demonstrates the potential for intelligent automation in space cybersecurity analysis. By leveraging local AI infrastructure and carefully engineered domain-specific prompts, the system provides comprehensive, consistent risk assessments while maintaining data sovereignty and operational independence.

The batch processing architecture ensures complete analysis coverage, addressing a critical limitation of manual assessment approaches. Integration with the existing Risk Assessment Tool Suite creates a hybrid capability that combines the speed and consistency of automated analysis with the nuanced judgment of human expertise.

Performance validation confirms the system's practical viability for operational deployment, with significant improvements in assessment speed and completeness compared to manual approaches. The modular architecture and standardized interfaces facilitate future enhancements and ensure long-term maintainability.

This work establishes a foundation for the next generation of intelligent cybersecurity tools in the space domain, demonstrating how AI can augment human expertise to address the growing complexity and scale of modern space missions.

% ============================================================================
% CHAPTER 9: DISCUSSION AND FUTURE WORK
% ============================================================================
\chapter{Discussion and Future Work}
\label{ch:discussion}

This chapter discusses the implications of the research findings, the technical contributions achieved, and the limitations identified during development. Additionally, it outlines directions for future work in the field of cybersecurity risk assessment for space projects.

\section{Research Contributions and Implications}

\subsection{Technical Contributions}

This research contributes to the field of space cybersecurity through several key technical achievements:

\begin{itemize}
    \item \textbf{Standardized Assessment Framework}: Development of a comprehensive, phase-specific framework for cybersecurity risk assessment throughout the space project lifecycle
    \item \textbf{Automated Tool Suite}: Implementation of integrated software tools that embody the framework principles and provide practical assessment capabilities
    \item \textbf{Asset Data Centralization}: Introduction of dynamic, CSV-based asset management that improves maintainability and consistency across assessment phases
    \item \textbf{Cross-Platform Compatibility}: Resolution of critical technical challenges including Unicode character encoding and executable environment compatibility
    \item \textbf{Professional User Experience}: Implementation of enterprise-grade user interfaces with comprehensive documentation and error handling
\end{itemize}

\subsection{Methodological Contributions}

The research advances cybersecurity risk assessment methodology in several important ways:

\begin{itemize}
    \item \textbf{Lifecycle Integration}: Establishment of assessment approaches tailored to specific project phases while maintaining methodological continuity
    \item \textbf{Space-Specific Adaptation}: Customization of traditional risk assessment methods to address the unique constraints and threat landscape of space systems
    \item \textbf{Standardization Achievement}: Creation of consistent assessment procedures enabling cross-project comparison and organizational learning
    \item \textbf{Practical Implementation}: Translation of theoretical frameworks into operational tools that can be readily adopted by space industry practitioners
\end{itemize}

\subsection{Industry Impact}

The developed framework and tools address critical industry needs:

\begin{itemize}
    \item \textbf{Efficiency Improvement}: Automation of repetitive assessment tasks reduces time and resource requirements
    \item \textbf{Consistency Enhancement}: Standardized approaches minimize assessment variability and improve reliability
    \item \textbf{Knowledge Capture}: Structured assessment processes facilitate capture and reuse of organizational cybersecurity expertise
    \item \textbf{Decision Support}: Comprehensive reporting capabilities provide actionable intelligence for cybersecurity investment decisions
\end{itemize}

\section{Technical Challenges and Solutions}

\subsection{Character Encoding Resolution}

A critical technical challenge encountered during development involved Unicode character compatibility in executable environments:

\begin{itemize}
    \item \textbf{Problem Identification}: Runtime errors occurred when executable files contained Unicode characters incompatible with Windows cp1252 encoding
    \item \textbf{Solution Implementation}: Systematic replacement of Unicode symbols with ASCII equivalents throughout all user-facing text
    \item \textbf{Process Development}: Creation of automated scanning procedures to identify and resolve encoding issues
    \item \textbf{Quality Assurance}: Implementation of testing protocols to prevent recurrence of encoding-related issues
\end{itemize}

\subsection{Asset Data Management Evolution}

The transition from static to dynamic asset loading represented a significant architectural improvement:

\begin{itemize}
    \item \textbf{Legacy Limitations}: Original implementation used hardcoded asset definitions leading to maintenance challenges
    \item \textbf{Centralization Benefits}: Migration to CSV-based asset loading improved consistency and reduced code duplication
    \item \textbf{Error Handling Enhancement}: Implementation of robust fallback mechanisms ensuring continued operation even when data files are unavailable
    \item \textbf{Scalability Achievement}: New architecture supports larger and more complex asset hierarchies without code modifications
\end{itemize}

\subsection{Executable Environment Compatibility}

Ensuring proper operation in compiled executable environments required careful attention to multiple technical aspects:

\begin{itemize}
    \item \textbf{Path Resolution Challenges}: Different execution contexts (development vs. executable) required robust file path handling mechanisms
    \item \textbf{Data File Packaging}: PyInstaller configuration required careful specification of data files and dependencies
    \item \textbf{Resource Management}: Proper allocation and cleanup of system resources in executable format
    \item \textbf{Testing Complexity}: Comprehensive testing across both development and deployed environments
\end{itemize}

\section{Limitations and Constraints}

\subsection{Scope Limitations}

The current research has several inherent limitations:

\begin{itemize}
    \item \textbf{Domain Specificity}: The framework is specifically designed for space systems and may require adaptation for other domains
    \item \textbf{Assessment Coverage}: While comprehensive for space applications, the framework does not address all possible emerging threats
    \item \textbf{Integration Complexity}: Full integration with existing enterprise risk management systems may require additional development
    \item \textbf{User Training Requirements}: Effective utilization requires understanding of cybersecurity principles and space system architectures
\end{itemize}

\subsection{Technical Constraints}

Several technical constraints limit the current implementation:

\begin{itemize}
    \item \textbf{Platform Dependencies}: Implementation relies on Python and specific library versions that may evolve over time
    \item \textbf{Scalability Boundaries}: Current architecture is optimized for single-user desktop operation rather than enterprise-scale deployment
    \item \textbf{Real-time Limitations}: Tools are designed for periodic assessment rather than continuous monitoring applications
    \item \textbf{Customization Constraints}: Framework adaptation for organization-specific requirements may require technical expertise
\end{itemize}

\subsection{Validation Limitations}

The validation process, while comprehensive from a technical perspective, has certain limitations:

\begin{itemize}
    \item \textbf{Case Study Scope}: Validation focused on technical functionality rather than large-scale organizational deployment
    \item \textbf{User Diversity}: Testing involved limited user diversity in terms of organizational context and expertise levels
    \item \textbf{Longitudinal Assessment}: Long-term effectiveness evaluation requires extended deployment periods
    \item \textbf{Comparative Analysis}: Limited comparison with alternative risk assessment approaches or tools
\end{itemize}

\section{Future Work Directions}

\subsection{Technical Enhancements}

Several technical improvement opportunities have been identified:

\begin{itemize}
    \item \textbf{Web-Based Interface}: Development of browser-based versions to improve accessibility and reduce deployment complexity
    \item \textbf{Advanced Analytics}: Integration of statistical analysis and data visualization capabilities for enhanced insight generation
    \item \textbf{API Development}: Creation of programmatic interfaces to support integration with external systems
    \item \textbf{Mobile Compatibility}: Adaptation for tablet and mobile devices to support field-based assessment activities
\end{itemize}

\subsection{Methodological Extensions}

The framework could be extended in several methodological directions:

\begin{itemize}
    \item \textbf{Quantitative Risk Modeling}: Integration of probabilistic risk models and Monte Carlo simulation capabilities
    \item \textbf{Machine Learning Integration}: Application of AI techniques for pattern recognition and predictive risk assessment
    \item \textbf{Real-time Monitoring}: Extension to support continuous monitoring and dynamic risk assessment updates
    \item \textbf{Multi-Domain Adaptation}: Modification for application in other critical infrastructure domains
\end{itemize}

\subsection{Industry Integration}

Future development could focus on enhanced industry integration:

\begin{itemize}
    \item \textbf{Standards Alignment}: Closer integration with evolving cybersecurity standards and regulatory requirements
    \item \textbf{Enterprise Integration}: Development of connectors for popular enterprise risk management and project management platforms
    \item \textbf{Collaborative Features}: Implementation of multi-user capabilities for distributed assessment teams
    \item \textbf{Training Materials}: Creation of comprehensive training programs and certification processes
\end{itemize}

\subsection{Research Extensions}

Additional research opportunities include:

\begin{itemize}
    \item \textbf{Effectiveness Studies}: Longitudinal studies of framework effectiveness in reducing cybersecurity incidents
    \item \textbf{Economic Analysis}: Cost-benefit analysis of implementing standardized cybersecurity risk assessment processes
    \item \textbf{User Experience Research}: Comprehensive usability studies across diverse organizational contexts
    \item \textbf{Comparative Evaluation}: Systematic comparison with alternative risk assessment methodologies and tools
\end{itemize}

% ============================================================================
% CHAPTER 10: CONCLUSION
% ============================================================================
\chapter{Conclusion}
\label{ch:conclusion}

This thesis has presented a comprehensive approach to cybersecurity risk assessment for space projects, addressing critical gaps in the standardization and automation of security evaluation processes. The research has produced both theoretical contributions through the development of a standardized risk assessment framework and practical contributions through the implementation of an integrated tool suite that embodies these principles.

\section{Research Achievements}

\subsection{Framework Development}

The research successfully developed a standardized framework for cybersecurity risk assessment that addresses the unique challenges of space systems throughout their project lifecycle. Key achievements include:

\begin{itemize}
    \item \textbf{Lifecycle Integration}: Creation of phase-specific assessment methodologies spanning from Business Initiation and Definition (BID) through operational phases
    \item \textbf{Space System Specialization}: Adaptation of traditional risk assessment approaches to address space-specific threats, assets, and operational constraints
    \item \textbf{Methodological Consistency}: Establishment of standardized approaches that enable consistent assessment practices across different projects and organizations
    \item \textbf{Practical Applicability}: Translation of theoretical concepts into operational procedures that can be readily implemented by space industry practitioners
\end{itemize}

\subsection{Tool Suite Implementation}

The practical embodiment of the framework through an integrated tool suite represents a significant technical achievement:

\begin{itemize}
    \item \textbf{Comprehensive Coverage}: Four specialized tools addressing different assessment phases and analytical requirements
    \item \textbf{Professional Quality}: Enterprise-grade user interfaces with consistent design patterns and comprehensive error handling
    \item \textbf{Technical Robustness}: Resolution of critical technical challenges including character encoding compatibility and cross-platform operation
    \item \textbf{Data Integration}: Implementation of centralized asset management and seamless data flow between assessment phases
\end{itemize}

\subsection{Technical Innovation}

The development process yielded several technical innovations that enhance the practical utility of cybersecurity risk assessment tools:

\begin{itemize}
    \item \textbf{Dynamic Asset Loading}: Transition from static asset definitions to flexible CSV-based asset management
    \item \textbf{Cross-Environment Compatibility}: Robust handling of different execution contexts from development to compiled executable deployment
    \item \textbf{Enhanced User Experience}: Implementation of comprehensive help systems and intuitive interface designs
    \item \textbf{Error Resilience}: Development of graceful error handling mechanisms that maintain operational continuity
\end{itemize}

\section{Contributions to Space Cybersecurity}

\subsection{Industry Impact}

This research addresses critical needs within the space industry:

\begin{itemize}
    \item \textbf{Standardization Achievement}: Provision of consistent methodologies that can be adopted across different organizations and projects
    \item \textbf{Efficiency Improvement}: Automation capabilities that reduce assessment time and resource requirements while improving consistency
    \item \textbf{Knowledge Systematization}: Structured approaches that facilitate capture and reuse of cybersecurity expertise
    \item \textbf{Decision Support Enhancement}: Comprehensive reporting and analysis capabilities that improve cybersecurity investment decisions
\end{itemize}

\subsection{Academic Contributions}

The research makes several important academic contributions:

\begin{itemize}
    \item \textbf{Theoretical Framework}: Development of structured approaches to space system cybersecurity risk assessment
    \item \textbf{Implementation Methodology}: Demonstration of how theoretical frameworks can be effectively translated into practical tools
    \item \textbf{Technical Problem Resolution}: Documentation of solutions to critical implementation challenges
    \item \textbf{Validation Approach}: Establishment of systematic validation methodologies for cybersecurity assessment tools
\end{itemize}

\section{Practical Significance}

\subsection{Immediate Applications}

The developed framework and tools provide immediate value to space industry practitioners:

\begin{itemize}
    \item \textbf{Assessment Standardization}: Ready-to-use tools that implement consistent assessment methodologies
    \item \textbf{Workflow Optimization}: Structured processes that improve assessment efficiency and completeness
    \item \textbf{Documentation Generation}: Automated reporting capabilities that reduce administrative burden
    \item \textbf{Knowledge Transfer}: Comprehensive documentation that facilitates adoption and training
\end{itemize}

\subsection{Strategic Value}

The research provides strategic value for organizational cybersecurity improvement:

\begin{itemize}
    \item \textbf{Risk Management Enhancement}: Systematic approaches that improve identification and mitigation of cybersecurity risks
    \item \textbf{Compliance Support}: Structured methodologies that support adherence to cybersecurity standards and regulations
    \item \textbf{Investment Optimization}: Evidence-based approaches that support optimal allocation of cybersecurity resources
    \item \textbf{Organizational Learning}: Knowledge capture mechanisms that preserve and propagate cybersecurity expertise
\end{itemize}

\section{Future Outlook}

\subsection{Evolution Potential}

The framework and tools are designed to evolve with advancing technology and changing threat landscapes:

\begin{itemize}
    \item \textbf{Extensibility}: Modular architecture that supports addition of new assessment capabilities
    \item \textbf{Adaptability}: Flexible design that can accommodate emerging threats and evolving space technologies
    \item \textbf{Integration Readiness}: Architecture that supports integration with advancing cybersecurity tools and platforms
    \item \textbf{Standards Alignment}: Design principles that facilitate alignment with evolving cybersecurity standards
\end{itemize}

\subsection{Research Continuity}

The research establishes a foundation for continued advancement in space cybersecurity:

\begin{itemize}
    \item \textbf{Methodological Framework}: Systematic approaches that can be extended and refined through future research
    \item \textbf{Technical Platform}: Software architecture that can serve as a basis for advanced cybersecurity tools
    \item \textbf{Validation Methodology}: Testing approaches that can be applied to future tool development efforts
    \item \textbf{Knowledge Base}: Documented solutions to technical challenges that inform future development projects
\end{itemize}

\section{Final Remarks}

The successful completion of this research demonstrates the feasibility and value of creating standardized, automated approaches to cybersecurity risk assessment for space systems. The combination of theoretical framework development with practical tool implementation provides a comprehensive solution that addresses real industry needs while establishing a foundation for future advancement.

The technical challenges overcome during development, particularly in areas of character encoding compatibility and asset data management, provide valuable lessons for future cybersecurity tool development efforts. The emphasis on professional user experience and comprehensive documentation ensures that the research contributions can be effectively utilized by practitioners across the space industry.

Looking forward, the modular and extensible architecture of both the framework and tools positions them to evolve with advancing technology and changing threat landscapes. This research thus contributes not only immediate practical value but also establishes a platform for continued innovation in space cybersecurity risk assessment.

In conclusion, this thesis demonstrates that systematic approaches to cybersecurity risk assessment, when properly implemented through professional-quality tools, can significantly enhance the security posture of space missions while reducing the burden on assessment practitioners.

% ============================================================================
% BIBLIOGRAPHY
% ============================================================================
\backmatter

\begin{thebibliography}{9}
\bibitem{NIST800-30}
National Institute of Standards and Technology (NIST) (2012) \emph{Guide for Conductiong Risk Assessments}
\bibitem{ISO27005}
ISO/IEC 27005 (2022) \emph{Information security, cybersecurity and privacy protection - Guidance on managing information security risks}
\bibitem{EBIOS}
Agence nationale de la sécurité des systèmes d'information (2024) \emph{La méthode EBIOS Risk Manager}
\bibitem{ITAR}
United States Department of State (1996) \emph{International Traffic in Arms Regulations (ITAR)}
\bibitem{ENISA Space Threat Landscape}
European Union Agency for Cybersecurity (ENISA) (2020) \emph{ENISA Threat Landscape for Space Systems}
\bibitem{Information Security for Space Systems}
Bundesamt für Sicherheit in der Informationstechnik (2023) \emph{Technical Guideline BSI TR-03184 Information Security for Space Systems}
\bibitem{CCSDS 350.1 G-3}
Consultative Committee for Space Data Systems (CCSDS) (2022) \emph{Security Threats Against Space Mission}
\bibitem{Space Threat Assessment}
Center for Strategic and International Studies (CSIS) (2023) \emph{Space Threat Assessment}
\bibitem{Cyber Security Toolkit}
UK Space Agency (2020) \emph{Cyber Security Toolkit}
\bibitem{Risk Management}
European Space Agency (ESA) (1999) \emph{Risk Management at ESA}
\bibitem{Attacks against Space Systems}
Ekzhin Ear, Jose L. C. Remy, Antonia Feffer, Shouhuai Xu (2023) \emph{Characterizing Cyber Attacks against Space Systems with Missing Data: Framework and Case Study}
\bibitem{Cybersecurity Analysis}
Vlad-Cosmin Matei (2021) \emph{Cybersecurity Analysis for the Internet-Connected Satellites}
\bibitem{Communication Security}
Pavur et al. (2020) \emph{A Tale of Sea and Sky On the Security of Maritime VSAT Communications}
\bibitem{Ground Segment Security}
Santamarta (2014) \emph{SATCOM Terminals: Hacking by Air, Sea, and Land}
%\bibitem{Supply Chain Risks}
%Peterson et al. (2019) \emph{Introduction to U-Net and Res-Net for Image Segmentation}
%\bibitem{Topic}
%Autore \emph{Titolo}
\end{thebibliography}


% ============================================================================
% ACKNOLEDGMENTS
% ============================================================================
\begin{acknowledgments}
Vorrei ringraziare la mia famiglia, i miei genitori e mia sorella Alessia per tutto il loro supporto, insieme a tutti gli altri parenti, nonni, zii e cugini. Un grazie al mio relatore e a tutti gli altri docenti da cui ho avuto modo di imparare moltissimo in questi anni. Un ringraziamento va anche agli amici di sempre, sempre pronti a supportarmi, grazie Alessandro, Alessio, Claudio, Gabriele e Massimo. Grazie anche a tutte le persone con cui ho avuto modo di stringere amicizia in questi anni di università: Adamaris, Claudia, Ester, Gioele, Giovanni, Niccolo, Saverio e Valerio. Un grazie anche al corpo bandistico di Jenne "Massimi Filiberto" nel quale suono, a tutti i suoi componenti e al maestro Marco.  Un grazie va anche a Phil Spencer e Jim Ryan per avermi tenuto compagnia nelle notti di esame con la loro sitcom. Spero di non aver dimenticato nessuno, nel caso mi scuso subito. Infine un grazie va anche a te che stai leggendo questa tesi.
\end{acknowledgments}

\end{document}